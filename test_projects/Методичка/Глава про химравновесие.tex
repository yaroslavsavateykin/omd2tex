\section{Равновесие в газах}

\textbf{Равновесием} в термодинамике называют такое состояние системы, при котором её параметры не зависят от времени и отсутствуют потоки массы и энергии. Если в системе протекает обратимая химическая реакция, то при химическом равновесии количества реагирующих веществ в реакционной смеси не изменяются с течением времени. Основная задача теории химического равновесия - расчёт равновесного состава химических систем. Для этого используют закон действующих масс (выражения для констант равновесия) и закон сохранения массы (и заряда).

\begin{equation*}
\fbox{\text{И}\text{с}х\text{о}\text{д}\text{н}\text{ы}\text{й} \text{с}\text{о}\text{с}\text{т}\text{а}\text{в}} \ce{->[K][\text{\text{з}\text{а}\text{к}\text{о}\text{н} \text{с}\text{о}х\text{р}\text{а}\text{н}\text{е}\text{н}\text{и}\text{я} \text{м}\text{а}\text{с}\text{с}\text{ы}}]} \fbox{\text{Р}\text{а}\text{в}\text{н}\text{о}\text{в}\text{е}\text{с}\text{н}\text{ы}\text{й} \text{с}\text{о}\text{с}\text{т}\text{а}\text{в}}
\end{equation*}

Каждому исходному составу химической системы соответствует только один равновесный состав. Это утверждение составляет суть математической теоремы, которая говорит о том, что при любых начальных условиях уравнения закона действующих масс имеют одно-единственное решение, удовлетворяющее химическим ограничениям (например, чтобы все концентрации были положительными).

\subsection{Константа равновесия}

Количественные расчёты химических равновесий основаны на втором законе термодинамики, точнее на следствии из него: при постоянных температуре и давлении в равновесных процессах энергия Гиббса не изменяется, т. е. $\Delta_r G = 0$. Это означает, что при химическом равновесии суммарная энергия Гиббса реагентов равна энергии Гиббса продуктов. 

\begin{equation}
\ce{aA + bB <=> cC + dD},
\label{eq:54eaa1}
\end{equation}

Пусть нашу систему в состоянии равновесия описывает уравнение реакции \cref{eq:54eaa1}. Если записать энергию Гиббса каждого газообразного компонента в виде функции, зависящей от температуры и давления \cref{eq:226f91}, а также энергию Гиббса реакции \cref{eq:cbc45e}, то можно получить уравнение изотермы химической реакции \cref{eq:b30f72} в газовой фазе. 

\begin{equation}
G(p,T) = G^{\circ}(T) + RT \ln \widetilde{p}
\label{eq:226f91}
\end{equation}

\begin{equation}
\Delta_r G = \sum_{i=1}^n \nu_i G_i
\label{eq:cbc45e}
\end{equation}

где $\nu_i$ - стехиометрический коэффициент в уравнении реакции, $G_i$ - энергия Гиббса $i$-ого компонента

\begin{equation}
\begin{aligned}
&\Delta_r G = cG_C + dG_D - aG_A - bG_B = &\\
&\quad = c G^\circ_C + d G^\circ_D - a G^\circ_A - b G^\circ_B + RT(c\ln \widetilde{p}_C + d\ln \widetilde{p}_D - a\ln \widetilde{p}_A - b\ln \widetilde{p}_B) = &\\
&\quad = \Delta_r G^\circ + RT \ln \frac{\widetilde{p}_C^c \widetilde{p}_D^d}{\widetilde{p}_A^a \widetilde{p}_B^b} &
\end{aligned}
\label{eq:b30f72}
\end{equation}

где $\Delta_r G^\circ$ - cтандартная энергия Гиббса реакции, равна энергии Гиббса реакции, парциальные давления участников которой - 1 бар.

\begin{equation}
K_p = \left[\frac{\widetilde{p}_C^c \widetilde{p}_D^d}{\widetilde{p}_A^a \widetilde{p}_B^b} \right]_{\text{\text{р}\text{а}\text{в}\text{н}.}}
\label{eq:74379b}
\end{equation}

\begin{equation}
K_C = \left[ \frac{\widetilde{C}_C^c \widetilde{C}_D^d}{\widetilde{C}_A^a \widetilde{C}_B^b} \right]_{\text{\text{р}\text{а}\text{в}\text{н}.}}
\label{eq:62767d}
\end{equation}

В этом выражении под знаком логарифма в квадратных скобках стоит произведение равновесных значений парциальных давлений продуктов и реагентов в степенях, равных стехиометрическим коэффициентам. Это выражение называют константой равновесия \cref{eq:74379b}, причём буква $p$ обозначает, что константа записана именно для газофазной реакции. Аналогично константу можно записать для реакции в растворе \cref{eq:62767d}.  Очень важно отметить, что под логарифмом стоит величина безразмерная, поскольку мы не можем взять логарифм от размерности. Для разрешения этой проблемы физхимики договорились использовать приведенные величины (обозначено волной над буквой) $\widetilde{p} = \frac{p}{p^\circ}, \widetilde{C} = \frac{C}{C^\circ}$. Здесь $p^\circ = 1 \text{ \text{б}\text{а}\text{р}}$ - стандартное давление, ${} C^\circ = 1 \text{ \text{м}\text{о}\text{л}\text{ь}/\text{л}}$. Поэтому при решении задач давления компонентов стоит выражать \textit{в барах}.

\noindent\rule{\textwidth}{0.5pt} %

\begin{itemize}\keepwithnext\itemsep0pt
    \item Константа реакции \textbf{зависит} от следующих факторов:
\end{itemize}

\begin{enumerate}\keepwithnext\itemsep0pt
    \setcounter{enumi}{0}
    \item \textit{Природа реагирующих веществ.}
    \setcounter{enumi}{1}
    \item \textit{Температура.} Для эндотермических реакций с повышением температуры константа равновесия будет увеличиваться, для экзотермических — уменьшаться.
\end{enumerate}

\begin{itemize}\keepwithnext\itemsep0pt
    \item Константа реакции \textbf{не зависит} от:
\end{itemize}

\begin{enumerate}\keepwithnext\itemsep0pt
    \setcounter{enumi}{0}
    \item \textit{Давление.} Давление может сместить положение равновесия в газовых системах (изменяя Q), но при данной температуре сама константа равновесия \textbf{не меняется}: для идеальных газов изменение внешнего давления не влияет на её численное значение.
    \setcounter{enumi}{1}
    \item \textit{Катализатор.} Присутствие катализатора в химически реагирующей системе \textbf{не влияет} на значение константы равновесия, так как состояние катализатора до и после реакции не изменяются.
    \setcounter{enumi}{2}
    \item \textit{Растворитель.} Растворитель \textbf{не оказывает влияния} на численное значение термодинамической константы равновесия, если он непосредственно не участвует в реакции. Однако при проведении одной и той же реакции в разных растворителях изменяется выход продуктов.
\end{enumerate}

\subsection{Изотерма реакции}

Проанализируем полученное уранение изотермы реакции \cref{eq:25b790}. Очень важно не путать энергию Гиббса реакции $\Delta_r G$ и стандартную энергию Гиббса $\Delta_r G^{\circ}$. При равновесии первая равна $0$, а вторая может принимать любые значения.

\begin{equation}
 \underbrace{   \Delta_r G }_{\begin{array}{c} {\text{\text{о}\text{т}\text{в}\text{е}\text{ч}\text{а}\text{е}\text{т} \text{з}\text{а}}}\\ {\text{\text{н}\text{а}\text{п}\text{р}\text{а}\text{в}\text{л}\text{е}\text{н}\text{и}\text{е} \text{п}\text{р}\text{о}\text{ц}\text{е}\text{с}\text{с}\text{а}}} \end{array}} = \underbrace{   \Delta_r G^\circ }_{\begin{array}{c} {\text{\text{о}\text{т}\text{в}\text{е}\text{ч}\text{а}\text{е}\text{т} \text{з}\text{а}}}\\ {\text{\text{р}\text{а}\text{в}\text{н}\text{о}\text{в}\text{е}\text{с}\text{н}\text{о}\text{е} \text{с}\text{о}\text{с}\text{т}\text{о}\text{я}\text{н}\text{и}\text{е}}} \end{array}} + \underbrace{   RT \ln \left( \frac{\tilde{p}_C^c \tilde{p}_D^d}{\tilde{p}_A^a \tilde{p}_B^b} \right) }_{\begin{array}{c} {\text{\text{о}\text{п}\text{р}\text{е}\text{д}\text{е}\text{л}\text{я}\text{е}\text{т} \text{в}\text{о}\text{з}\text{м}\text{о}\text{ж}\text{н}\text{о}\text{с}\text{т}\text{ь}}}\\ {\text{\text{и}\text{з}\text{м}\text{е}\text{н}\text{и}\text{т}\text{ь} \text{н}\text{а}\text{п}\text{р}\text{а}\text{в}\text{л}\text{е}\text{н}\text{и}\text{е} \text{п}\text{р}\text{о}\text{ц}\text{е}\text{с}\text{с}\text{а}}} \end{array}} 
\label{eq:25b790}
\end{equation}

При достижении равновесия $\Delta_r G = 0$, тогда стандартная энергия Гиббса реакции \cref{eq:d0be42}

\begin{equation}
\Delta_r G^{\circ} = -RT \ln \left[ \frac{\tilde{p}^c_C \tilde{p}^d_D}{\tilde{p}^a_A \tilde{p}^b_B} \right]_{\text{\text{р}\text{а}\text{в}\text{н}}}
\label{eq:d0be42}
\end{equation}

Тогда константу реакции можно записать в виде \cref{eq:a11dd9}

\begin{equation}
K_p = \left[ \frac{\tilde{p}^c_C \tilde{p}^d_D}{\tilde{p}^a_A \tilde{p}^b_B} \right]_{\text{\text{р}\text{а}\text{в}\text{н}}} = \exp \left( -\frac{\Delta_r G^{\circ}}{RT} \right)
\label{eq:a11dd9}
\end{equation}

С учётом константы равновесия условие химического равновесия и уравнение изотермы можно переписать в виде \cref{eq:2eeaef}  при $\Delta_r G = 0$ и \cref{eq:36a1a4} при $\Delta_r G \neq 0$.

\begin{equation}
\Delta_r G^{\circ} = -RT \ln K_p
\label{eq:2eeaef}
\end{equation}

Внимательнее присмотримся к уравнению \cref{eq:2eeaef}:

\begin{enumerate}\keepwithnext
    \setcounter{enumi}{0}
    \item Если $\Delta_r G^\circ \gg 0$, то $K_P \ll 1$: это означает, что в системе мало продуктов и много исходных веществ, т. е. равновесие сильно смещено влево, прямая реакция практически не идёт;
    \setcounter{enumi}{1}
    \item Если $\Delta_r G^\circ \ll 0$, то $K_P \gg 1$: в этом случае в равновесной системе преобладают продукты и прямая реакция практически необратима;
    \setcounter{enumi}{2}
    \item Если $\Delta_r G^\circ$ не сильно отличается от 0, то константа равновесия $K_P$ по порядку величины близка к 1, а в равновесной системе реагенты и продукты присутствуют в сопоставимых количествах.
\end{enumerate}

\begin{equation}
\Delta_r G = -RT \ln K_p + RT \ln \left(\frac{\tilde{p}^c_C \tilde{p}^d_D}{\tilde{p}^a_A \tilde{p}^b_B}\right)
\label{eq:36a1a4}
\end{equation}

Стоит отметить, что в записи уравнения \cref{eq:d0be42} давления в дроби под знаком логарифма $\tilde{p}_i$ - это парциальные давления давления компонентов именно в \ul{равновесном состоянии}, поскольку $\Delta_r G = 0$, а в уравнении \cref{eq:36a1a4} $\Delta_r G \neq 0$, поэтому там давления отвечают другому состоянию. Чтобы избежать путаницы, обычно отношение давлений в \cref{eq:36a1a4} обозначают буквой $Q$ от слова \textit{quotient} (отношение):

\begin{equation}
\Delta_r G = -RT \ln K_p + RT \ln Q = RT \ln \left( \frac{Q}{K_p} \right)
\label{eq:505b9a}
\end{equation}

Выражение \cref{eq:505b9a} позволяет определить направление реакции при любом составе смеси:

\begin{enumerate}\keepwithnext
    \setcounter{enumi}{0}
    \item Если $Q > K_p$, то продуктов реакции образовалось больше, чем при равновесии. В этом случае $\Delta_rG > 0$ и реакция протекает справа налево, т. е. в обратном направлении. 
    \setcounter{enumi}{1}
    \item Если $Q < K_p$, то продуктов реакции меньше, чем надо для равновесия. При этом составе смеси $\Delta_rG < 0$ и реакция протекает в прямом направлении, слева направо. 
    \setcounter{enumi}{2}
    \item Если $Q = K_p$, то $\Delta_rG = 0$ и имеет место химическое равновесие.
\end{enumerate}

\noindent\rule{\textwidth}{0.5pt} %

Эти выражения играют очень важную роль в прикладной термодинамике:  

\begin{enumerate}\keepwithnext
    \setcounter{enumi}{0}
    \item Используя справочные данные для расчёта $\Delta_r G^{\circ}$, можно, не проводя эксперимента, рассчитать равновесный состав смеси;  
    \setcounter{enumi}{1}
    \item Если известны стандартная энергия Гиббса реакции (или константа равновесия) и парциальные давления реагирующих веществ, можно по знаку $\Delta_r G$ судить о направлении процесса.  
\end{enumerate}

\subsection{Пересчет константы равновесия на новую температуру}

Для того чтобы найти температурную зависимость константы равновесия, необходимо продиффернцировать уравнение \cref{eq:2eeaef} при постоянном давлении:

\begin{equation*}
\frac{\partial}{\partial T} \left(\Delta_r G^{\circ} \right)= \frac{\partial}{\partial T}(-RT \ln K_p)
\end{equation*}

Поскольку $K_p$  - зависит от температуры ($K_p  =f(T)$), то справа имеем сложную функцию, и по правилу дифференцирования сложной функции получаем:

\begin{equation*}
\left(\frac{\partial \Delta_r G^\circ}{\partial T} \right)_p = -R \ln K_p -RT \left(\frac{\partial \ln K_p}{\partial T} \right)_p 
\end{equation*}

Группируем и получаем:

\begin{equation}
\left(\frac{\partial \ln K_p}{\partial T} \right)_p= \frac{1}{R} \left( \frac{\Delta_r G^\circ}{T^2} - \frac{1}{T} \left(\frac{\partial \Delta_r G^\circ}{\partial T}\right)_p\right)
\label{eq:8f58a6}
\end{equation}

Запишем уравнение Гиббса-Гельмгольца \cref{eq:6ae2b5} для $\Delta_r G^\circ$ и подставим его в \cref{eq:8f58a6}, считая, что энтальпия и энтропия не зависят от температуры:

\begin{equation}
\Delta_r G^\circ = \Delta_r H^\circ - T \Delta_r S^\circ
\label{eq:6ae2b5}
\end{equation}

\begin{equation}
\left(\frac{\partial \ln K_p}{\partial T} \right)_p= \frac{1}{R} \left( \frac{\Delta_r H^\circ - T \Delta_r S^\circ}{T^2} - \frac{1}{T} (-\Delta_r S^\circ )\right) = \frac{\Delta_r H^\circ}{RT^2}
\label{eq:b71bb1}
\end{equation}

Аналогично для постоянного давления получаем:

\begin{equation}
\left(\frac{\partial \ln K_C}{\partial T} \right)_V= \frac{\Delta_r U^\circ}{RT^2}
\label{eq:b58ef2}
\end{equation}

Уравнения \cref{eq:b71bb1} и \cref{eq:b58ef2} называют уравнениями изобары и изохоры реакции соответственно. Из этих уравнений следует, что влияние температуры на константу равновесия определяется знаком теплового эффекта.

\begin{enumerate}\keepwithnext\itemsep0pt
    \setcounter{enumi}{0}
    \item Если реакция эндотермическая, т. е. $\Delta_r H^\circ > 0$, то $\left( \frac{\partial \ln K_p}{\partial T} \right)_p > 0$ и с повышением температуры константа равновесия будет расти, равновесие сместится в сторону продуктов реакции.
    \setcounter{enumi}{1}
    \item Если реакция экзотермическая, т. е. $\Delta_r H^\circ < 0$, то $\left( \frac{\partial \ln K_p}{\partial T} \right)_p < 0$ и с повышением температуры константа равновесия будет уменьшаться, равновесие сместится в сторону реагентов.
\end{enumerate}

Эти качественные выводы о влиянии температуры на химическое равновесие согласуются с общим принципом смещения равновесия (\textbf{принципом Ле Шателье-Брауна}): 

\begin{quote}\slshape\noindent
Если на систему, находящуюся в равновесии, оказать внешнее воздействие, то равновесие сместится так, чтобы уменьшить эффект произведенного воздействия.
\end{quote}

Иначе:

\begin{itemize}\keepwithnext\itemsep0pt
    \item Повышение (или понижение) температуры сдвигает равновесие в сторону реакции, протекающей с поглощением (выделением) теплоты.
    \item Повышение давления сдвигает равновесие в сторону уменьшения количества молекул газа.
    \item Добавление в равновесную смесь какого-либо компонента реакции сдвигает равновесие в сторону уменьшения количества этого компонента.
\end{itemize}

При интегрировании уравнений \cref{eq:b71bb1} и \cref{eq:b58ef2} нужно знать температурные зависимости $\Delta_r H^\circ(T)$ и $\Delta_r U^\circ(T)$. Если расчеты равновесий проводятся в небольшом температурном интервале, можно принять, что эти величины постоянны. Тогда:

\begin{equation*}
\ln K_p = -\frac{\Delta_r H^\circ}{RT} + \text{const} = \frac{A}{T} + B
\end{equation*}

\begin{equation*}
\ln \frac{K_p(T_2)}{K_p(T_1)} = \frac{\Delta_r H^\circ}{R} \cdot \left( \frac{1}{T_1} - \frac{1}{T_2} \right)
\end{equation*}

\begin{equation*}
\ln K_c = -\frac{\Delta_r U^\circ}{RT} + \text{const} = \frac{A_1}{T} + B_1
\end{equation*}

\begin{equation*}
\ln \frac{K_c(T_2)}{K_c(T_1)} = \frac{\Delta_r U^\circ}{R} \cdot \left( \frac{1}{T_1} - \frac{1}{T_2} \right)
\end{equation*}

где $A_i$, $B_i$ — некоторые параметры, определяемые при статистической обработке экспериментальных данных.

\subsection{Расчет равновесного состава}

При расчетах равновесного состава принимают, что химические реакции протекают при $T = \text{const}$, поэтому для определения выхода продуктов необходимо знать энергии Гиббса участников реакции при заданной температуре. Если эти величины известны, задача расчета равновесий сводится к решению уравнения или системы уравнений различной сложности. 

\begin{enumerate}\keepwithnext\itemsep0pt
    \setcounter{enumi}{0}
    \item \textit{Химические равновесия в газах: реакции без изменения числа молекул}
\end{enumerate}

Рассмотрим реакцию между идеальными газами, в которой сумма стехиометрических коэффициентов в правой и левой частях уравнения одинакова. Состояние системы в начальный момент времени и по достижении равновесия можно схематично представить следующим образом:

\begin{equation*}
A + B \rightleftarrows C + D
\end{equation*}

\begin{equation*}
\begin{matrix} & A & + & B & = & C & + & D & \text{\text{с}\text{у}\text{м}\text{м}\text{а}} \\[6pt] t=0 & a & & b & & 0 & & 0 & a + b \\[6pt] t_{\text{\text{р}\text{а}\text{в}\text{н}}} & a - \xi & & b - \xi & & \xi & & \xi & a + b \\[6pt] x_i & \dfrac{a-\xi}{a+b} & & \dfrac{b-\xi}{a+b} & & \dfrac{\xi}{a+b} & & \dfrac{\xi}{a+b} & 1 \\[6pt] \tilde p_i & \dfrac{a-\xi}{a+b}\,\tilde p & & \dfrac{b-\xi}{a+b}\,\tilde p & & \dfrac{\xi}{a+b}\,\tilde p & & \dfrac{\xi}{a+b}\,\tilde p & \tilde p \end{matrix}
\end{equation*}

где $\tilde{p}$ — приведенное суммарное давление (бар), $\xi$ — количество прореагировавших веществ, $x_i$ — мольная доля i-го компонента, $\tilde{p}_i$ — парциальное давление

Константу равновесия этой реакции записывают в виде:

\begin{equation*}
K_p = \frac{\tilde{p}_C\tilde{p}_D}{\tilde{p}_A\tilde{p}_B} = \frac{\xi^2}{(a-\xi)(b-\xi)}
\end{equation*}

\begin{enumerate}\keepwithnext\itemsep0pt
    \setcounter{enumi}{1}
    \item \textit{Химические равновесия в газах: реакции с изменением числа молекул}
\end{enumerate}

Рассмотрим реакцию между идеальными газами, для которой сумма стехиометрических коэффициентов различна:

\begin{equation*}
A + B \rightleftarrows C
\end{equation*}

\begin{equation*}
\begin{matrix} & A & + & B & = & C & \text{\text{с}\text{у}\text{м}\text{м}\text{а}} \\[6pt] t=0 & a & & b & & 0 & a + b \\[6pt] t_{\text{\text{р}\text{а}\text{в}\text{н}}} & a - \xi & & b - \xi & & \xi & a + b - \xi \\[6pt] x_i & \dfrac{a-\xi}{\,a+b-\xi\,} & & \dfrac{b-\xi}{\,a+b-\xi\,} & & \dfrac{\xi}{\,a+b-\xi\,} & 1 \\[6pt] \tilde p_i & \dfrac{a-\xi}{\,a+b-\xi\,}\,\tilde p & & \dfrac{b-\xi}{\,a+b-\xi\,}\,\tilde p & & \dfrac{\xi}{\,a+b-\xi\,}\,\tilde p & \tilde p \end{matrix}
\end{equation*}

Константа равновесия этой реакции  

\begin{equation*}
K_p = \frac{\tilde{p}_C}{\tilde{p}_A\tilde{p}_B} = \frac{\xi(a + b - \xi)}{(a - \xi)(b - \xi)} \cdot \frac{1}{\tilde{p}}
\end{equation*}

Решая полученное уравнение, находят значение $\xi$ и равновесный состав смеси. При заданной температуре константа равновесия есть величина постоянная, поэтому в рассматриваемом случае выход продукта зависит от общего давления. Для рассмотренной реакции с ростом $\tilde{p}$ увеличивается значение $\xi$.  

Полученный результат согласуется с принципом Ле Шателье-Брауна: возрастание давления должно приводить к смещению равновесия в сторону веществ, занимающих меньший объем.  

Введение в систему инертного газа при $p = \text{const}$ эквивалентно уменьшению общего давления:  

\begin{itemize}\keepwithnext\itemsep0pt
    \item Для реакций с уменьшением числа газообразных веществ $\rightarrow$ равновесие смещается в сторону исходных веществ  
    \item Для реакций с увеличением количества молей $\rightarrow$ равновесие смещается в сторону продуктов реакции
\end{itemize}

test \cite{еремин2013основы} \cite{еремин2007теоретическая} 