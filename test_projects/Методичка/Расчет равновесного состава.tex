\subsection{Расчет равновесного состава}

При расчетах равновесного состава принимают, что химические реакции протекают при $T = \text{const}$, поэтому для определения выхода продуктов необходимо знать энергии Гиббса участников реакции при заданной температуре. Если эти величины известны, задача расчета равновесий сводится к решению уравнения или системы уравнений различной сложности. 

\begin{enumerate}\keepwithnext\itemsep0pt
    \setcounter{enumi}{0}
    \item \textit{Химические равновесия в газах: реакции без изменения числа молекул}
\end{enumerate}

Рассмотрим реакцию между идеальными газами, в которой сумма стехиометрических коэффициентов в правой и левой частях уравнения одинакова. Состояние системы в начальный момент времени и по достижении равновесия можно схематично представить следующим образом:

\begin{equation*}
A + B \rightleftarrows C + D
\end{equation*}

\begin{equation*}
\begin{matrix} & A & + & B & = & C & + & D & \text{\text{с}\text{у}\text{м}\text{м}\text{а}} \\[6pt] t=0 & a & & b & & 0 & & 0 & a + b \\[6pt] t_{\text{\text{р}\text{а}\text{в}\text{н}}} & a - \xi & & b - \xi & & \xi & & \xi & a + b \\[6pt] x_i & \dfrac{a-\xi}{a+b} & & \dfrac{b-\xi}{a+b} & & \dfrac{\xi}{a+b} & & \dfrac{\xi}{a+b} & 1 \\[6pt] \tilde p_i & \dfrac{a-\xi}{a+b}\,\tilde p & & \dfrac{b-\xi}{a+b}\,\tilde p & & \dfrac{\xi}{a+b}\,\tilde p & & \dfrac{\xi}{a+b}\,\tilde p & \tilde p \end{matrix}
\end{equation*}

где $\tilde{p}$ — приведенное суммарное давление (бар), $\xi$ — количество прореагировавших веществ, $x_i$ — мольная доля i-го компонента, $\tilde{p}_i$ — парциальное давление

Константу равновесия этой реакции записывают в виде:

\begin{equation*}
K_p = \frac{\tilde{p}_C\tilde{p}_D}{\tilde{p}_A\tilde{p}_B} = \frac{\xi^2}{(a-\xi)(b-\xi)}
\end{equation*}

\begin{enumerate}\keepwithnext\itemsep0pt
    \setcounter{enumi}{1}
    \item \textit{Химические равновесия в газах: реакции с изменением числа молекул}
\end{enumerate}

Рассмотрим реакцию между идеальными газами, для которой сумма стехиометрических коэффициентов различна:

\begin{equation*}
A + B \rightleftarrows C
\end{equation*}

\begin{equation*}
\begin{matrix} & A & + & B & = & C & \text{\text{с}\text{у}\text{м}\text{м}\text{а}} \\[6pt] t=0 & a & & b & & 0 & a + b \\[6pt] t_{\text{\text{р}\text{а}\text{в}\text{н}}} & a - \xi & & b - \xi & & \xi & a + b - \xi \\[6pt] x_i & \dfrac{a-\xi}{\,a+b-\xi\,} & & \dfrac{b-\xi}{\,a+b-\xi\,} & & \dfrac{\xi}{\,a+b-\xi\,} & 1 \\[6pt] \tilde p_i & \dfrac{a-\xi}{\,a+b-\xi\,}\,\tilde p & & \dfrac{b-\xi}{\,a+b-\xi\,}\,\tilde p & & \dfrac{\xi}{\,a+b-\xi\,}\,\tilde p & \tilde p \end{matrix}
\end{equation*}

Константа равновесия этой реакции  

\begin{equation*}
K_p = \frac{\tilde{p}_C}{\tilde{p}_A\tilde{p}_B} = \frac{\xi(a + b - \xi)}{(a - \xi)(b - \xi)} \cdot \frac{1}{\tilde{p}}
\end{equation*}

Решая полученное уравнение, находят значение $\xi$ и равновесный состав смеси. При заданной температуре константа равновесия есть величина постоянная, поэтому в рассматриваемом случае выход продукта зависит от общего давления. Для рассмотренной реакции с ростом $\tilde{p}$ увеличивается значение $\xi$.  

Полученный результат согласуется с принципом Ле Шателье-Брауна: возрастание давления должно приводить к смещению равновесия в сторону веществ, занимающих меньший объем.  

Введение в систему инертного газа при $p = \text{const}$ эквивалентно уменьшению общего давления:  

\begin{itemize}\keepwithnext\itemsep0pt
    \item Для реакций с уменьшением числа газообразных веществ $\rightarrow$ равновесие смещается в сторону исходных веществ  
    \item Для реакций с увеличением количества молей $\rightarrow$ равновесие смещается в сторону продуктов реакции
\end{itemize}

test \cite{еремин2013основы} \cite{еремин2007теоретическая} 