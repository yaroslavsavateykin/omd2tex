\subsection{Энтропия}

Некоторые явления происходят самопроизвольно, для осуществления других требуются определенные усилия. В мире есть нечто, определяющее направление самопроизвольных процессов - изменений, для которых не требуется совершение работы. Однако ни один из процессов не происходит самопроизвольно, для осуществления любого из них необходимо произвести работу.  

Существование двух типов процессов - самопроизвольных и вынужденных - обобщается вторым законом термодинамики, который мы рассмотрим позднее.  

Итак, с помощью первого закона термодинамики и понятия внутренней энергии можно определить все допустимые превращения системы. Чтобы определить, какие из них будут происходить самопроизвольно, необходимо использовать второй закон термодинамики и понятие энтропии.  

В дальнейшем будет показано, что энтропия является мерой молекулярного беспорядка в системе. Она дает возможность оценить, можно ли перейти из некоторого состояния в другое состояние путем самопроизвольного превращения.  

Существует несколько эквивалентных формулировок второго закона термодинамики. Однако для решения практических задач в химии, чаще используют обобщенную формулировку второго закона, данную Гиббсом и Гуггенгеймом:  

\begin{quote}\slshape\noindent
«Существует экстенсивная функция состояния термодинамической системы — энтропия (S). При протекании в изолированной системе обратимых процессов эта функция остается неизменной, а при необратимых — увеличивается.»  
\end{quote}

\begin{equation*}
dS \geq 0
\end{equation*}

Важно: в состоянии термодинамического равновесия энтропия изолированной системы достигает максимума (dS = 0).  

\textit{Энтропия} — свойство термодинамической системы (ее внутренняя переменная), поэтому, согласно постулатам термодинамики, при равновесии она является функцией температуры (или внутренней энергии) и внешних переменных.  

В изолированной системе знак изменения энтропии является критерием направленности самопроизвольного процесса:  

\begin{enumerate}\keepwithnext
    \setcounter{enumi}{0}
    \item если $\Delta$S = 0 (S = S$_{max}$, энтропия достигла своего максимального значения), то система находится в состоянии термодинамического равновесия;  
    \setcounter{enumi}{1}
    \item если $\Delta$S > 0 (S $\rightarrow$S$_{max}$, энтропия возрастает), то процесс самопроизвольно протекает в прямом направлении, т.е. термодинамически возможен;  
    \setcounter{enumi}{2}
    \item если $\Delta$S  < 0 (S $\rightarrow$S$_{min}$, энтропия убывает), то самопроизвольно протекать может лишь обратный процесс, связанный с увеличением энтролии, прямой процесс термодинамически невозможен.  
\end{enumerate}

\subsubsection{Термодинамическое определение энтропии}

Под термодинамическим определением энтропии понимают ее изменение dS в результате физического или химического превращения. Это основано на том, что глубина протекания процесса зависит от количества энергии, переданной в форме теплоты. Работа не влияет на энтропию, так как она связана с упорядоченным движением молекул, а следовательно не изменяет степень беспорядка.  

Термодинамическое определение энтропии выражается соотношением:  

\begin{equation*}
dS = \frac{dq_{\text{о}\text{к}\text{р}}}{T}
\end{equation*}

В соответствии с приведенной выше формулой, теплота выражается в джоулях (Дж), а температура в кельвинах (К), поэтому единица измерения энтропии Дж/К. Размерность энтропии в применении к 1 моль вещества: \[S\] = Дж/(моль $\cdot$ К), т.е. в тех же единицах, что и универсальная газовая постоянная R и мольная теплоемкость.  

\subsubsection{Неравенство Клаузиуса}

Рассмотрим формулировку второго закона термодинамики, данную Э.Гуггенгеймом, которая базируется на термодинамическом определении энтропии:  

\begin{quote}\slshape\noindent
«Существует функция состояния энтропия S, которая обладает следующими свойствами: если при бесконечно малом изменении состояния системы поглощенную из окружающей среды теплоту обозначить через ${} {d} {q}_{ {\text{о}\text{к}\text{р}}}$, то:   
для самопроизвольных процессов: ${dS >}\frac{ {d} {q}_{ {\text{о}\text{к}\text{р}}}}{ {T}}$,   
для вынужденных процессов: ${dS <}\frac{ {d} {q}_{ {\text{о}\text{к}\text{р}}}}{ {T}}$,   
для любых обратимых процессов: ${dS =}\frac{ {d} {q}_{ {\text{о}\text{к}\text{р}}}}{ {T}}$.»  
\end{quote}

Объединив три записанных выше соотношения в одно можно получить самую распространенную математическую запись второго закона термодинамики, называемую неравенством Клаузиуса:  

\begin{equation*}
dS \geq \frac{dq_{\text{о}\text{к}\text{р}}}{T}.
\end{equation*}