\subsection{Расчет изменения энтропии для различных процессов}

\begin{enumerate}\keepwithnext
    \setcounter{enumi}{0}
    \item Нагревание или охлаждение при постоянном давлении (объеме).  
\end{enumerate}

Количество теплоты, необходимое для изменения температуры системы при p = const, выражают с помощью теплоемкости: $\delta$Q$_{\text{о}\text{б}\text{р}}$ = C$_{p}$dT. C учетом соотношения:  

\begin{equation*}
{\Delta}S = \int_{T_{1}}^{T_{2}}{\frac{\delta Q_{\text{о}\text{б}\text{р}}}{T} = \int_{T_{1}}^{T_{2}}{\frac{C_{p}}{T}dT.}}
\end{equation*}

Если теплоемкость не зависит от температуры в интервале от Т$_{1}$ до Т$_{2}$, то уравнение, представленное выше, можно проинтегрировать и представить в виде:  

\begin{equation*}
{\Delta}S = C_{p}\ln\frac{T_{2}}{T_{1}}.
\end{equation*}

\begin{enumerate}\keepwithnext
    \setcounter{enumi}{1}
    \item Изотермическое расширение или сжатие.  
\end{enumerate}

Для расчета энтропии в этом случае надо знать уравнение состояния системы. Расчет основан на использовании соотношений Максвелла.  

\begin{equation*}
{\Delta}S = \int_{V_{1}}^{V_{2}}{(\frac{\partial S}{\partial V})_{T}dV = \int_{V_{1}}^{V_{2}}{(\frac{\partial p}{\partial T})_{V}dV,}}
\end{equation*}

\begin{equation*}
{\Delta}S = \int_{p_{1}}^{p_{2}}{(\frac{\partial S}{\partial p})_{T}dp = - \int_{V_{1}}^{V_{2}}{(\frac{\partial V}{\partial T})_{p}dp.}}
\end{equation*}

В частности, для изотермического расширения идеального газа (p = nRT /V) интегральные формулы преобразуются к виду:  

\begin{equation*}
{\Delta}S = nRln\frac{V_{2}}{V_{1}} = - nRln\frac{p_{2}}{p_{1}}.
\end{equation*}

\begin{enumerate}\keepwithnext
    \setcounter{enumi}{2}
    \item Фазовые переходы.  
\end{enumerate}

При обратимом фазовом переходе температура остается постоянной, а теплота фазового перехода при постоянном давлении равна $\Delta$$_{\text{ф}.\text{п}.}$H, поэтому изменение энтропии равно:  

\begin{equation*}
{\Delta}S = \frac{1}{T}\int_{}^{}{\delta Q_{\text{ф}.\text{п}.} = \frac{{\Delta}_{\text{ф}.\text{п}.}H}{T_{\text{ф}.\text{п}.}}.}
\end{equation*}

При плавлении и кипении теплота поглощается, поэтому энтропия в этих процессах возрастает: S$_{\text{т}\text{в}}$  < S$_{\text{ж}}$  < S$_{\text{г}}$. При этом энтропия окружающей среды уменьшается на величину $\Delta$$_{\text{ф}.\text{п}.}$S, поэтому изменение энтропии вселенной равно нулю, как и полагается для обратимого процесса в изолированной системе.  

\ul{Уровень I}  

\ul{Задача 1}  

Найдите $\Delta$$_{r}$G$^{\circ}$$_{298}$ для реакции при Т = 298К и определите, идет ли реакция самопроизвольно:  

\begin{equation*}
CH_{4} + H_{2}O \rightarrow CO + 3H_{2}
\end{equation*}

Если известно, что $\Delta$$_{f}$H$^{\circ}$$_{298}$(CH$_{4}$) = -74,8 кДж/моль, $\Delta$$_{f}$H$^{\circ}$$_{298}$(H$_{2}$O) = -241,8 кДж/моль, $\Delta$$_{f}$H$^{\circ}$$_{298}$(CO) = -110,5 кДж/моль, S$^{\circ}$$_{298}$(CH$_{4}$) = 186,3 Дж/моль$\cdot$К, S$^{\circ}$$_{298}$(H$_{2}$O) = 188,7 Дж/моль$\cdot$К, S$^{\circ}$$_{298}$(CO) = 197,5 Дж/моль$\cdot$К, S$^{\circ}$$_{298}$(H$_{2}$) = 130,5 Дж/моль$\cdot$К.  

\ul{Ответ:} 142,3 кДж/моль  

\ul{Решение 1}  

Определим $\Delta$$_{r}$H$^{\circ}$$_{298}$ для указанной реакции:  

$\Delta$$_{r}$H$^{\circ}$$_{298}$ = 206,1 кДж/моль  

Аналогично найдем $\Delta$$_{r}$S$^{\circ}$$_{298}$:  

$\Delta$$_{r}$S$^{\circ}$$_{298}$ = 214 Дж/моль$\cdot$К  

Согласно уравнению $\Delta$G = $\Delta$H - T$\Delta$S, посчитаем энергию Гиббса для реакции:  

$\Delta$$_{r}$G$^{\circ}$$_{298}$ = 142,3 кДж/моль  

Получается, $\Delta$$_{r}$G$^{\circ}$$_{298}$ > 0, следовательно, реакция самопроизвольно не протекает  

\ul{Задача 2}  

Процесс Боша-Габера используют в промышленности для синтеза аммиака из простых веществ. Вычислите энтропию происходящей реакции при температуре 773К. Справочные данные: $\Delta$$_{r}$S$^{\circ}$$_{773}$(N$_{2 (\text{г}\text{а}\text{з})}$) = 191,53 Дж/моль$\cdot$К, $\Delta$$_{r}$S$^{\circ}$$_{773}$(Н$_{2 (\text{г}\text{а}\text{з})}$) = 130,55 Дж/моль$\cdot$К, $\Delta$$_{r}$S$^{\circ}$$_{773}$(NH$_{3 (\text{г}\text{а}\text{з})}$) = 192,63 Дж/моль$\cdot$К  

\ul{Ответ:} -197,92 Дж/моль$\cdot$К  

\ul{Решение 2}  

Запишем уравнение протекающей реакции:  

\begin{equation*}
N_{2(\text{г}\text{а}\text{з})} + \ 3\text{Н}_{2(\text{г}\text{а}\text{з})} \rightleftarrows 2NH_{3(\text{г}\text{а}\text{з})}
\end{equation*}

\begin{equation*}
{\Delta}_{r}S = \sum_{}^{}{\theta_{i}S_{i(\text{к}\text{о}\text{н})}}\  - \ \sum_{}^{}{\theta_{j}S_{j(\text{и}\text{с}х)}}
\end{equation*}

Используя данную формулу, произведем расчет энтропии промышленного процесса:  

$\Delta$$_{r}$S$^{\circ}$$_{773}$ = 2 $\cdot$ 192,63 - 191,53 - 3 $\cdot$ 130,55 = -197,92 кДж/моль$\cdot$К  

\ul{Задача 3}  

Рассмотрим обратимую реакцию димеризации оксида азота (IV):  

\begin{equation*}
2{NO}_{2(\text{г})} \rightleftarrows N_{2}O_{4(\text{г})}
\end{equation*}

Термодинамические характеристики представлены в таблице:  

\begingroup
\centering
\begin{longtblr}[ ]{colspec={Q[c]Q[c]Q[c]}, hlines,vlines}
 Вещество                        &  $\Delta$$_{f}$H$^{\circ}$$_{298}$, кДж/моль  &  $\Delta$S$^{\circ}$$_{298}$, Дж/моль$\cdot$К  \\
 NO$_{2 (\text{г})}$              &  33,5                                     &  240,2                         \\
 N$_{2}$O$_{4 (\text{г})}$  &  9,6                                      &  303,8                         \\

\end{longtblr}
\endgroup

Определите энтальпию, энтропию и энергию Гиббса при Т = 298К для данной реакции.  

\ul{Ответ:} -57,4 кДж/моль; -176,6 Дж/моль$\cdot$К; -4773,2 Дж/моль  

\ul{Решение 3}  

Энтальпия реакции:  

$\Delta$$_{r}$H$^{\circ}$$_{298}$ = 9,6 - 2 $\cdot$ 33,5 = -57,4 кДж/моль  

Энтропия реакции:  

$\Delta$S$^{\circ}$$_{298}$ = 303,8 - 2 $\cdot$ 240,2 = -176,6 Дж/моль$\cdot$К  

Энергия Гиббса реакции:  

$\Delta$$_{r}$G$^{\circ}$$_{298}$ = $\Delta$H - T$\Delta$S = -57400 - 298 $\cdot$ (-176,6) = -4773,2 Дж/моль  

\ul{Задача 4}  

Известно, что для реакции образования йодоводорода из простых веществ $\Delta$G$^{\circ}$$_{298}$ = -1780 Дж/моль, $\Delta$H$^{\circ}$$_{298}$ = 26,48 кДж/моль, $\Delta$S$^{\circ}$$_{298}$ (HI) = 206,48 Дж/моль$\cdot$К, $\Delta$S$^{\circ}$$_{298}$ (H$_{2}$) = 130,52 Дж/моль$\cdot$К. Определите значение $\Delta$S$^{\circ}$$_{298}$ (I$_{2}$).  

\ul{Ответ:} 116,67 Дж/моль$\cdot$К  

\ul{Решение 4}  

Запишем реакцию образования йодоводорода из простых веществ:  

\begin{equation*}
H_{2} + I_{2} \rightarrow 2HI
\end{equation*}

Определим энтропию реакции:  

$\Delta$$_{r}$G$^{\circ}$$_{298}$ = $\Delta$H$^{\circ}$$_{298}$ - T$\Delta$S$^{\circ}$$_{298}$  

2$\cdot$ 1780 = 2 $\cdot$ 26480 - 298 $\cdot$ $\Delta$S$^{\circ}$$_{298}$  

$\Delta$S$^{\circ}$$_{298}$ = 165,77 Дж/моль$\cdot$К  

Рассчитаем энтропию для йода, используя закон Гесса:  

2 $\cdot$ $\Delta$S$^{\circ}$$_{298}$ (HI) - $\Delta$S$^{\circ}$$_{298}$ (H$_{2}$) - $\Delta$S$^{\circ}$$_{298}$ (I$_{2}$) = $\Delta$S$^{\circ}$$_{298}$  

2 $\cdot$ 206,48 - 130,52 - $\Delta$S$^{\circ}$$_{298}$ (I$_{2}$) = 165,77  

Отсюда $\Delta$S$^{\circ}$$_{298}$ (I$_{2}$) = 116,67 Дж/моль$\cdot$К.  