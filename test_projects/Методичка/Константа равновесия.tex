\subsection{Константа равновесия}

Количественные расчёты химических равновесий основаны на втором законе термодинамики, точнее на следствии из него: при постоянных температуре и давлении в равновесных процессах энергия Гиббса не изменяется, т. е. $\Delta_r G = 0$. Это означает, что при химическом равновесии суммарная энергия Гиббса реагентов равна энергии Гиббса продуктов. 

\begin{equation}
\ce{aA + bB <=> cC + dD},
\label{eq:54eaa1}
\end{equation}

Пусть нашу систему в состоянии равновесия описывает уравнение реакции \cref{eq:54eaa1}. Если записать энергию Гиббса каждого газообразного компонента в виде функции, зависящей от температуры и давления \cref{eq:226f91}, а также энергию Гиббса реакции \cref{eq:cbc45e}, то можно получить уравнение изотермы химической реакции \cref{eq:b30f72} в газовой фазе. 

\begin{equation}
G(p,T) = G^{\circ}(T) + RT \ln \widetilde{p}
\label{eq:226f91}
\end{equation}

\begin{equation}
\Delta_r G = \sum_{i=1}^n \nu_i G_i
\label{eq:cbc45e}
\end{equation}

где $\nu_i$ - стехиометрический коэффициент в уравнении реакции, $G_i$ - энергия Гиббса $i$-ого компонента

\begin{equation}
\begin{aligned}
&\Delta_r G = cG_C + dG_D - aG_A - bG_B = &\\
&\quad = c G^\circ_C + d G^\circ_D - a G^\circ_A - b G^\circ_B + RT(c\ln \widetilde{p}_C + d\ln \widetilde{p}_D - a\ln \widetilde{p}_A - b\ln \widetilde{p}_B) = &\\
&\quad = \Delta_r G^\circ + RT \ln \frac{\widetilde{p}_C^c \widetilde{p}_D^d}{\widetilde{p}_A^a \widetilde{p}_B^b} &
\end{aligned}
\label{eq:b30f72}
\end{equation}

где $\Delta_r G^\circ$ - cтандартная энергия Гиббса реакции, равна энергии Гиббса реакции, парциальные давления участников которой - 1 бар.

\begin{equation}
K_p = \left[\frac{\widetilde{p}_C^c \widetilde{p}_D^d}{\widetilde{p}_A^a \widetilde{p}_B^b} \right]_{\text{\text{р}\text{а}\text{в}\text{н}.}}
\label{eq:74379b}
\end{equation}

\begin{equation}
K_C = \left[ \frac{\widetilde{C}_C^c \widetilde{C}_D^d}{\widetilde{C}_A^a \widetilde{C}_B^b} \right]_{\text{\text{р}\text{а}\text{в}\text{н}.}}
\label{eq:62767d}
\end{equation}

В этом выражении под знаком логарифма в квадратных скобках стоит произведение равновесных значений парциальных давлений продуктов и реагентов в степенях, равных стехиометрическим коэффициентам. Это выражение называют константой равновесия \cref{eq:74379b}, причём буква $p$ обозначает, что константа записана именно для газофазной реакции. Аналогично константу можно записать для реакции в растворе \cref{eq:62767d}.  Очень важно отметить, что под логарифмом стоит величина безразмерная, поскольку мы не можем взять логарифм от размерности. Для разрешения этой проблемы физхимики договорились использовать приведенные величины (обозначено волной над буквой) $\widetilde{p} = \frac{p}{p^\circ}, \widetilde{C} = \frac{C}{C^\circ}$. Здесь $p^\circ = 1 \text{ \text{б}\text{а}\text{р}}$ - стандартное давление, ${} C^\circ = 1 \text{ \text{м}\text{о}\text{л}\text{ь}/\text{л}}$. Поэтому при решении задач давления компонентов стоит выражать \textit{в барах}.

\noindent\rule{\textwidth}{0.5pt} %

\begin{itemize}\keepwithnext\itemsep0pt
    \item Константа реакции \textbf{зависит} от следующих факторов:
\end{itemize}

\begin{enumerate}\keepwithnext\itemsep0pt
    \setcounter{enumi}{0}
    \item \textit{Природа реагирующих веществ.}
    \setcounter{enumi}{1}
    \item \textit{Температура.} Для эндотермических реакций с повышением температуры константа равновесия будет увеличиваться, для экзотермических — уменьшаться.
\end{enumerate}

\begin{itemize}\keepwithnext\itemsep0pt
    \item Константа реакции \textbf{не зависит} от:
\end{itemize}

\begin{enumerate}\keepwithnext\itemsep0pt
    \setcounter{enumi}{0}
    \item \textit{Давление.} Давление может сместить положение равновесия в газовых системах (изменяя Q), но при данной температуре сама константа равновесия \textbf{не меняется}: для идеальных газов изменение внешнего давления не влияет на её численное значение.
    \setcounter{enumi}{1}
    \item \textit{Катализатор.} Присутствие катализатора в химически реагирующей системе \textbf{не влияет} на значение константы равновесия, так как состояние катализатора до и после реакции не изменяются.
    \setcounter{enumi}{2}
    \item \textit{Растворитель.} Растворитель \textbf{не оказывает влияния} на численное значение термодинамической константы равновесия, если он непосредственно не участвует в реакции. Однако при проведении одной и той же реакции в разных растворителях изменяется выход продуктов.
\end{enumerate}