\subsubsection{Закон Гесса}

Вернемся к закону Гесса, упомянутому ранее, и рассмотрим его подробнее. Данный закон имеет большое практическое значение, так как позволяет рассчитывать теплоты химических реакций, которые не могут быть измерены экспериментально или это измерение сопряжено с большими трудностями.  

Закон Гесса, упомянутый ранее, имеет большое практическое значение, так как он позволяет рассчитывать тепловые эффекты многих химических реакций, которые не могут быть измерены экспериментально, или это измерение сопряжено с большими трудностями. Особенно удобно проводить такие расчеты, пользуясь следующими следствиями из закона Гесса.  

\begin{quote}\slshape\noindent
Следствие 1. Тепловой эффект реакции равен сумме энтальпий образования продуктов за вычетом суммы энтальпий образования исходных веществ с учетом стехиометрических коэффициентов \cref{fig:a43184}.  
\end{quote}

\begin{equation*}
\Delta H\  = \ \sum_{}^{}{ni{\Delta}_{f}Hi}\  - \ \sum_{}^{}{nj{\Delta}_{f}Hj},
\end{equation*}

где $\Delta$H — изменение энтальпии соответствующей реакции, $\Delta$$_{f}$Hi и $\Delta$$_{f}$Hj — энтальпии образования продуктов реакции и исходных веществ соответственно.  

\begin{figure}[H] 
\centering
\includegraphics[width = \textwidth, keepaspectratio]{./images/image2v.png}
\caption{Схематическая иллюстрация 1 следствия закона Гесса.}
\label{fig:a43184}
\end{figure}

\begin{quote}\slshape\noindent
Следствие 2. Тепловой эффект реакции равен сумме энтальпий сгорания исходных веществ за вычетом суммы энтальпий сгорания продуктов с учетом стехиометрических коэффициентов.  
\end{quote}

\begin{equation*}
{\Delta}_{r}H\  = \ \sum_{}^{}{n_i{\Delta}_{f}H_i}\  - \ \sum_{}^{}{n_j{\Delta}_{f}H_j},
\end{equation*}

где $\Delta$H — изменение энтальпии соответствующей реакции, $\Delta$$_{c}$Hi и $\Delta$$_{c}$Hj — энтальпии сгорания исходных веществ и продуктов реакции соответственно.  

Это следствие обычно используют для расчета тепловых эффектов реакций органических соединений. Иногда в качестве следствия закона Гесса рассматривают еще одно утверждение.  

\begin{quote}\slshape\noindent
Следствие 3. Тепловой эффект реакции равен сумме энергий разрываемых химических связей за вычетом суммы энергий образующихся связей.  
\end{quote}

Энергией связи A-B называют энергию, необходимую для разрыва связи и разведения образующихся частиц на бесконечное расстояние:  

\begin{equation*}
 AB_{(\text{г})} \rightarrow A_{(\text{г})} + B_{(\text{г})}. 
\end{equation*}

\begin{equation*}
{\Delta}_{r}H\  = \ \sum_{}^{}{ni {E}_{ {\text{с}\text{в}\ (\text{р}\text{а}\text{з}\text{р}\text{ы}\text{в})}}}\  - \ \sum_{}^{}{nj {E}_{ {\text{с}\text{в}\ (\text{о}\text{б}\text{р})}}}.
\end{equation*}

\begin{figure}[H] 
\centering
\includegraphics[width = \textwidth, keepaspectratio]{./images/image3v.png}
\caption{Схематическая иллюстрация 3 следствия закона Гесса.}

\end{figure}

Важно: энергия связи всегда положительна.  

Непосредственно из закона Гесса следует, что тепловые эффекты двух взаимно обратных реакций равны по модулю и обратны по знаку:  

\begin{equation*}
{\Delta}\text{Н}(\text{А} \rightarrow \text{В}) = - {\Delta}\text{Н}(\text{В} \rightarrow \text{А})
\end{equation*}

Чтобы нагляднее понять суть закона Гесса, рассмотрим схему:  

\begin{figure}[H] 
\centering
\includegraphics[width = \textwidth, keepaspectratio]{./images/image4v.png}


\end{figure}

Из которой следует, что:  

\begin{equation*}
{\Delta}\text{Н}_{1} + {\Delta}\text{Н}_{2} + {\Delta}\text{Н}_{3} = {\Delta}\text{Н}_{4} + {\Delta}\text{Н}_{5}.
\end{equation*}

Определим энтальпию представленного ниже процесса, зная энтальпии образования веществ, участвующих в нем:  

\begin{equation*}
CO + H_{2}O \rightarrow CO_{2} + H_{2} + {\Delta}H_{1}.
\end{equation*}

Распишем подробнее реакции образования веществ, из которых мы сможем составить процесс:  

\begin{equation*}
C + 0,5O_{2} \rightarrow CO + {\Delta}H_{2};
\end{equation*}

\begin{equation*}
C + O_{2} \rightarrow CO_{2} + {\Delta}H_{3};
\end{equation*}

\begin{equation*}
H_{2} + 0,5O_{2} \rightarrow H_{2}O + {\Delta}H_{4}.
\end{equation*}

Группируя упомянутые выше реакции, получаем:  

\begin{equation*}
{\Delta}H_{1} = \ {\Delta}H_{3} - {\Delta}H_{2} - {\Delta}H_{4}.
\end{equation*}

Решим другую задачу, определим энтальпию образования угарного газа:  

\begin{equation*}
C + O_{2} \rightarrow CO_{2} - 393,5\ \text{к}\text{Д}\text{ж}/\text{м}\text{о}\text{л}\text{ь};
\end{equation*}

\begin{equation*}
C + 0,5O_{2} \rightarrow CO + {\Delta}H_{1};
\end{equation*}

\begin{equation*}
CO + 0,5O_{2} \rightarrow CO_{2} - 283\ \text{к}\text{Д}\text{ж}/\text{м}\text{о}\text{л}\text{ь}.
\end{equation*}

Сгруппировав приведенные ниже реакции, получаем:  

\begin{equation*}
- 393,5 - {\Delta}H_{1} = - 283.
\end{equation*}

${\Delta}H_{1}$= -110,5 кДж/моль.  