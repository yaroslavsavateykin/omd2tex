\subsection{Основные понятия}

Основная физическая величина в термодинамике - это работа. Это форма передачи энергии термодинамической системе при изменении ее внешних параметров (например, объема). Пример совершения работы - расширение газа, который толкает поршень и поднимает определенный груз. В качестве одного из видов работы в химической термодинамике рассматривается механическая работа расширения (W>0) или сжатия (W <0) системы реагирующих веществ при постоянном внешнем давлении. Помимо этого в термодинамике существует еще несколько базовых ключевых понятий:  

\begin{enumerate}\keepwithnext
    \setcounter{enumi}{0}
    \item \textit{Энергия системы} (Е) — это способность системы совершать работу. Она может передаваться от одной системы к другой в виде работы (упорядоченной формы) и теплоты (неупорядоченной формы; посредством хаотического движения молекул). Существует два типа химических реакций, различающихся по тепловому эффекту:  
\end{enumerate}

\begin{itemize}\keepwithnext
    \item Экзотермические (от греч. е'хо — вне, снаружи)  
\end{itemize}

\begin{quote}\slshape\noindent
 Химические реакции, идущие с выделением теплоты. Система нагревается и  отдает энергию в окружающую среду (Q > 0). К экзотермическим процессам относятся все реакции горения. 
\end{quote}

\begin{itemize}\keepwithnext
    \item Эндотермические (от греч. e'ndon — внутри)  
\end{itemize}

\begin{quote}\slshape\noindent
 Химические реакции, идущие с поглощением теплоты. Система охлаждается  и забирает энергию из окружающей среды (Q  < 0). Примером таког  процесса является испарение воды  
\end{quote}

Важно: при размещении системы в вакууме теплообмен отсутствует.  

\begin{enumerate}\keepwithnext
    \setcounter{enumi}{1}
    \item \textit{Внутренняя энергия системы} (U) - это сумма кинетических и потенциальных энергий молекул этой системы. Она является функцией состояния и зависит от текущего состояния системы, а не от способа достижения этого состояния. К примеру, при изменении давления в системе значение внутренней энергии также меняется. В термодинамике полная энергия системы называется ее внутренней энергией.  
\end{enumerate}

В отличие от внутренней энергии, теплота и работа не являются функциями состояния, они зависят от вида процесса. Об изменении внутренней энергии в процессе судят по количеству переданной теплоты и количеству совершенной работы. Единицы измерения теплоты и работы совпадают с единицами измерения энергии. В СИ [Q] =[W] =Дж или кратные ему единицы.  

\begin{enumerate}\keepwithnext
    \setcounter{enumi}{2}
    \item \textit{Изменением внутренней энергии} ($\Delta$U) является разность энергии конечного состояния \cref{eq:b1ebdc} и начального состояния \cref{eq:ba09d1}:  
\end{enumerate}

\begin{equation}
 \Delta U = U_{2} - U_{1}
\label{eq:ba09d1}
\end{equation}

Предположим, что система в результате поглощения теплоты Q переходит из состояния 1 в состояние 2. В общем случае эта теплота расходуется на изменение внутренней энергии системы $\Delta$U и на совершение работы W против внешних сил:  

\begin{equation}
Q = \Delta U + W
\label{eq:b1ebdc}
\end{equation}

Уравнение \cref{eq:b1ebdc} выражает закон сохранения энергии. Так, если теплота сообщается газу в цилиндре, закрытом поршнем, газ, во-первых, нагревается, т.е. его внутренняя энергия U увеличивается, а во-вторых, расширяется, т.е. производит работу W (подъем поршня). Закон сохранения энергии в форме \cref{eq:b1ebdc} называют первым законом термодинамики.  

Для элементарных процессов с бесконечно малыми изменениями параметров \cref{eq:b1ebdc} принимает вид:  

\begin{equation}
\delta Q = dU + \delta W = dU + pdV + \delta W^{'}
\label{eq:2ef6e2}
\end{equation}

где pdV — работа расширения системы; $\delta$W' - сумма других видов работ (электрической, сил поверхностного натяжения и др.). Внутренняя энергия соответствует функции состояния системы, поэтому перед символом U поставлен знак полного дифференциала (d). Теплота и работа не являются функциями состояния, и их бесконечно малые количества обозначены буквой $\delta$.  

Уравнения \cref{eq:b1ebdc} и \cref{eq:2ef6e2} описывают первый закон термодинамики в интегральной и дифференциальной формах соответственно. В термомеханических системах при протекании процессов совершается только работа расширения или сжатия, т.е. $\delta$W'= 0. Тогда \cref{eq:2ef6e2} принимает вид:  

\begin{equation}
\delta Q = dU + pdV
\label{eq:67db69}
\end{equation}

Рассмотрим, используя уравнения \cref{eq:2ef6e2} и \cref{eq:67db69}, применение первого закона термодинамики к так называемым изопроцессам в идеальном газе.  

\begin{enumerate}\keepwithnext
    \setcounter{enumi}{0}
    \item Изотермический процесс (T = const). В идеальном газе силы межмолекулярного взаимодействия равны нулю. Внутренняя энергия идеального газа зависит от его температуры, количества вещества и не зависит от давления и объема, поэтому для данных условий U = const, dU = 2. Уравнение \cref{eq:67db69} имеет вид:  
\end{enumerate}

\begin{equation}
\delta Q_{T}\  = \ \delta W = \ pdV
\label{eq:66b8fb}
\end{equation}

\begin{equation*}
Q_{T}\  = W.
\end{equation*}

Теплота, сообщенная системе, в изотермическом процессе полностью расходуется на совершение работы расширения. Для 1 моль идеального газа:  

\begin{equation*}
p = \frac{RT}{V}.
\end{equation*}

Подставив значение давления в \cref{eq:66b8fb} и проинтегрировав, получим выражение для работы изотермического расширения 1 моль идеального газа \cref{eq:a4d5a3}

\begin{equation}
W = RTln\frac{V_{2}}{V_{1}} = RTln\frac{p_{1}}{p_{2}}
\label{eq:a4d5a3}
\end{equation}

так как по закону Бойля —Мариотта (pV)$_{T}$ = const.  

\begin{enumerate}\keepwithnext
    \setcounter{enumi}{1}
    \item Изохорный процесс (V = const). При постоянном объеме dV = 0, значит, работа расширения газа $\delta$W = pdV = 0. Уравнение \cref{eq:67db69} принимает вид:  
\end{enumerate}

\begin{equation*}
 \delta Q_{V}\  = \ dU; 
\end{equation*}

\begin{equation*}
 Q_{V}\  = \ U_{2} - U_{1} = \ \Delta U. 
\end{equation*}

В изохорном процессе теплота, сообщенная системе, полностью расходуется на увеличение ее внутренней энергии и характеризует изменение состояния системы.  

\begin{enumerate}\keepwithnext
    \setcounter{enumi}{2}
    \item Изобарный процесс (р = const). Постоянную величину р можно внести под знак дифференциала, поэтому работа расширения $\delta$W = pdV = d(pV). Тогда \cref{eq:67db69} имеет вид:  
\end{enumerate}

\begin{equation*}
\delta Q_{p} = dU + \ d(pV) = d(U + pV) = dH;
\end{equation*}

\begin{equation*}
\delta Q_{p} = H_{2} - H_{1} = \Delta H.
\end{equation*}

Величины pV и U характеризуют состояние системы. Их сумма H = U + pV также соответствует функции состояния, которую называют энтальпией (от греч. enthálpo — нагреваю).  

В изобарном процессе теплота, сообщенная системе, расходуется на увеличение ее внутренней энергии и на совершение работы расширения против сил внешнего давления и характеризует изменение состояния системы. Работа расширения идеального газа от объема V$_{1}$ до объема V$_{2}$ в равновесном изобарном процессе определяется уравнением:  

\begin{equation*}
W = \int_{V_{1}}^{V_{2}}{pdV = p(V_{2} - V_{1}).}
\end{equation*}

Поговорим подробнее о новом понятии, которое не встречалось нам ранее, - об энтальпии. Энтальпия показывает, сколько теплоты содержится в системе при определенном давлении и объеме. При постоянном давлении изменение энтальпии равно количеству теплоты, которое выделяется или поглощается в процессе (при отсутствии всех видов работ, кроме работы расширения).  

Важно: энтальпия противоположна по знаку тепловому эффекту, $\Delta$H = -Q.  

Рассмотрим закон Гесса, который поможет объяснить нам, как производить термохимические расчеты реакций. Суть закона Гесса:  

\begin{quote}\slshape\noindent
«Теплота химической реакции при постоянном объеме или давлении (тепловой эффект химической реакции) не зависит от способа проведения процесса, а определяется только состоянием реагентов и продуктов реакции при условии, что единственной работой, совершаемой системой, является работа расширения.»  
\end{quote}

При расчете тепловых эффектов реакций принимают T = const. Для обозначения изменения любой термодинамической функции при протекании химической реакции пишут $\Delta_r$ (r означает «reaction»), например $\Delta_rU$ , $\Delta_rH$. Теплоту химической реакции при постоянном давлении иногда называют просто энтальпией реакции и записывают, как $\Delta_rH$. В термохимии, в отличие от других приложений термодинамики, теплота считается положительной, если она выделяется в окружающую среду, и отрицательной, если поглощается. Это означает, что в случае экзотермической реакции $\Delta_rH$  < 0, а для эндотермического процесса $\Delta_rH$ > 0.  

Основная практическая значимость закона, который был сформулирован Г. И. Гессом в следующем виде:  

\begin{quote}\slshape\noindent
«когда образуется какое-либо химическое соединение, то при этом всегда выделяется одно и то же количество тепла независимо от того, происходит ли образование этого соединения непосредственно или же косвенным путем и в несколько приемов», 
\end{quote}

заключается в том, что он позволяет оперировать с термохимическими уравнениями, как с алгебраическими. Важно отметить, что такие операции должны проводиться в одинаковых (стандартизованных) условиях. Введем несколько новых понятий:  

\begin{itemize}\keepwithnext
    \item \textit{Стандартной энтальпией реакции} $\Delta_rH_T^\circ$ называют энтальпию реакции между веществами, находящимися в стандартных состояниях при температуре T.  
    \item \textit{Стандартная энтальпия образования вещества} $\Delta_fH_T^\circ$ (f означает «formation») — называют энтальпию (теплоту) образования 1 моль этого вещества из простых веществ в стандартных (наиболее стабильных) состояниях при стандартных условиях. Согласно этому определению, энтальпия (теплота) образования простого вещества при стандартных условиях равна нулю при любой температуре. Понятие «энтальпия образования» используют не только для обычных веществ, но и для ионов в растворе. При этом за точку отсчета принят ион Н$^{+}$, для которого стандартная энтальпия образования в водном растворе также полагается равной нулю: $\Delta$$_f$H$_{\text{Т}}$ (H$^{+}$) = 0.  
    \item Стандартная энтальпия сгорания $\Delta_cH_T^\circ$ (c — «combustion») — называют энтальпию реакции полного окисления газообразным кислородом при p(O$_{2}$) = 1 бар одного моля вещества.  
\end{itemize}