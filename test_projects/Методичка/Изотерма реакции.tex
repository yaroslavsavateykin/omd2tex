\subsection{Изотерма реакции}

Проанализируем полученное уранение изотермы реакции \cref{eq:25b790}. Очень важно не путать энергию Гиббса реакции $\Delta_r G$ и стандартную энергию Гиббса $\Delta_r G^{\circ}$. При равновесии первая равна $0$, а вторая может принимать любые значения.

\begin{equation}
 \underbrace{   \Delta_r G }_{\begin{array}{c} {\text{\text{о}\text{т}\text{в}\text{е}\text{ч}\text{а}\text{е}\text{т} \text{з}\text{а}}}\\ {\text{\text{н}\text{а}\text{п}\text{р}\text{а}\text{в}\text{л}\text{е}\text{н}\text{и}\text{е} \text{п}\text{р}\text{о}\text{ц}\text{е}\text{с}\text{с}\text{а}}} \end{array}} = \underbrace{   \Delta_r G^\circ }_{\begin{array}{c} {\text{\text{о}\text{т}\text{в}\text{е}\text{ч}\text{а}\text{е}\text{т} \text{з}\text{а}}}\\ {\text{\text{р}\text{а}\text{в}\text{н}\text{о}\text{в}\text{е}\text{с}\text{н}\text{о}\text{е} \text{с}\text{о}\text{с}\text{т}\text{о}\text{я}\text{н}\text{и}\text{е}}} \end{array}} + \underbrace{   RT \ln \left( \frac{\tilde{p}_C^c \tilde{p}_D^d}{\tilde{p}_A^a \tilde{p}_B^b} \right) }_{\begin{array}{c} {\text{\text{о}\text{п}\text{р}\text{е}\text{д}\text{е}\text{л}\text{я}\text{е}\text{т} \text{в}\text{о}\text{з}\text{м}\text{о}\text{ж}\text{н}\text{о}\text{с}\text{т}\text{ь}}}\\ {\text{\text{и}\text{з}\text{м}\text{е}\text{н}\text{и}\text{т}\text{ь} \text{н}\text{а}\text{п}\text{р}\text{а}\text{в}\text{л}\text{е}\text{н}\text{и}\text{е} \text{п}\text{р}\text{о}\text{ц}\text{е}\text{с}\text{с}\text{а}}} \end{array}} 
\label{eq:25b790}
\end{equation}

При достижении равновесия $\Delta_r G = 0$, тогда стандартная энергия Гиббса реакции \cref{eq:d0be42}

\begin{equation}
\Delta_r G^{\circ} = -RT \ln \left[ \frac{\tilde{p}^c_C \tilde{p}^d_D}{\tilde{p}^a_A \tilde{p}^b_B} \right]_{\text{\text{р}\text{а}\text{в}\text{н}}}
\label{eq:d0be42}
\end{equation}

Тогда константу реакции можно записать в виде \cref{eq:a11dd9}

\begin{equation}
K_p = \left[ \frac{\tilde{p}^c_C \tilde{p}^d_D}{\tilde{p}^a_A \tilde{p}^b_B} \right]_{\text{\text{р}\text{а}\text{в}\text{н}}} = \exp \left( -\frac{\Delta_r G^{\circ}}{RT} \right)
\label{eq:a11dd9}
\end{equation}

С учётом константы равновесия условие химического равновесия и уравнение изотермы можно переписать в виде \cref{eq:2eeaef}  при $\Delta_r G = 0$ и \cref{eq:36a1a4} при $\Delta_r G \neq 0$.

\begin{equation}
\Delta_r G^{\circ} = -RT \ln K_p
\label{eq:2eeaef}
\end{equation}

Внимательнее присмотримся к уравнению \cref{eq:2eeaef}:

\begin{enumerate}\keepwithnext
    \setcounter{enumi}{0}
    \item Если $\Delta_r G^\circ \gg 0$, то $K_P \ll 1$: это означает, что в системе мало продуктов и много исходных веществ, т. е. равновесие сильно смещено влево, прямая реакция практически не идёт;
    \setcounter{enumi}{1}
    \item Если $\Delta_r G^\circ \ll 0$, то $K_P \gg 1$: в этом случае в равновесной системе преобладают продукты и прямая реакция практически необратима;
    \setcounter{enumi}{2}
    \item Если $\Delta_r G^\circ$ не сильно отличается от 0, то константа равновесия $K_P$ по порядку величины близка к 1, а в равновесной системе реагенты и продукты присутствуют в сопоставимых количествах.
\end{enumerate}

\begin{equation}
\Delta_r G = -RT \ln K_p + RT \ln \left(\frac{\tilde{p}^c_C \tilde{p}^d_D}{\tilde{p}^a_A \tilde{p}^b_B}\right)
\label{eq:36a1a4}
\end{equation}

Стоит отметить, что в записи уравнения \cref{eq:d0be42} давления в дроби под знаком логарифма $\tilde{p}_i$ - это парциальные давления давления компонентов именно в \ul{равновесном состоянии}, поскольку $\Delta_r G = 0$, а в уравнении \cref{eq:36a1a4} $\Delta_r G \neq 0$, поэтому там давления отвечают другому состоянию. Чтобы избежать путаницы, обычно отношение давлений в \cref{eq:36a1a4} обозначают буквой $Q$ от слова \textit{quotient} (отношение):

\begin{equation}
\Delta_r G = -RT \ln K_p + RT \ln Q = RT \ln \left( \frac{Q}{K_p} \right)
\label{eq:505b9a}
\end{equation}

Выражение \cref{eq:505b9a} позволяет определить направление реакции при любом составе смеси:

\begin{enumerate}\keepwithnext
    \setcounter{enumi}{0}
    \item Если $Q > K_p$, то продуктов реакции образовалось больше, чем при равновесии. В этом случае $\Delta_rG > 0$ и реакция протекает справа налево, т. е. в обратном направлении. 
    \setcounter{enumi}{1}
    \item Если $Q < K_p$, то продуктов реакции меньше, чем надо для равновесия. При этом составе смеси $\Delta_rG < 0$ и реакция протекает в прямом направлении, слева направо. 
    \setcounter{enumi}{2}
    \item Если $Q = K_p$, то $\Delta_rG = 0$ и имеет место химическое равновесие.
\end{enumerate}

\noindent\rule{\textwidth}{0.5pt} %

Эти выражения играют очень важную роль в прикладной термодинамике:  

\begin{enumerate}\keepwithnext
    \setcounter{enumi}{0}
    \item Используя справочные данные для расчёта $\Delta_r G^{\circ}$, можно, не проводя эксперимента, рассчитать равновесный состав смеси;  
    \setcounter{enumi}{1}
    \item Если известны стандартная энергия Гиббса реакции (или константа равновесия) и парциальные давления реагирующих веществ, можно по знаку $\Delta_r G$ судить о направлении процесса.  
\end{enumerate}