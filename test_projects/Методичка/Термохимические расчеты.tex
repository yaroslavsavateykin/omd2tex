\subsection{Термохимические расчеты}

Большинство химических реакций происходит при постоянном давлении, поэтому энергетический (тепловой) эффект реакции оценивают изменением энтальпии. Уравнение реакции, где указано изменение энтальпии $\Delta$H или тепловой эффект $Q_p$, называют термохимическим.  

Чтобы можно было сравнивать тепловые эффекты различных реакций, термохимические расчеты обычно относят к 1 моль вещества и условиям, принятым за стандартные. Стандартные условия — это давление 10$^{5}$ Па (100 кПа, или 1 бар) и произвольная температура. Стандартные тепловые эффекты при постоянном давлении принято обозначать $\Delta$H$^{\text{о}}$. Большинство справочных данных по тепловым эффектам приводят для температуры 25 $^{\circ}$С, или 298 К.  

В термохимических уравнениях указывают также агрегатное и фазовое состояния и аллотропную модификацию реагентов и продуктов: г — газ, ж — жидкость, т — твердое вещество; S(ромб), S(монокл), C(графит), C(алмаз) и т. д. Следует отметить, что в термохимическом уравнении стехиометрические коэффициенты показывают не только соотношение между реагентами и продуктами реакции, но и реальные количества веществ (в молях или кмолях). Стехиометрические коэффициенты в термохимических уравнениях могут быть и дробными.  

\begin{quote}\slshape\noindent
Важно: теплоты образования вещества в разных агрегатных состояниях различаются.  
\end{quote}

Приведенное выше определение стандартных условий надо дополнить еще одним определением:  

Стандартное состояние вещества — наиболее устойчивое состояние при стандартных условиях. Например: графит, ромбическая сера, белый фосфор, кислород O$_{2}$ (но не озон О$_{3}$), а также газообразный хлор (не жидкий или твердый), жидкий бром (не газообразный или твердый) и т.п. Стандартное состояние может быть отнесено к любой температуре.  

\subsubsection{Закон Гесса}

Вернемся к закону Гесса, упомянутому ранее, и рассмотрим его подробнее. Данный закон имеет большое практическое значение, так как позволяет рассчитывать теплоты химических реакций, которые не могут быть измерены экспериментально или это измерение сопряжено с большими трудностями.  

Закон Гесса, упомянутый ранее, имеет большое практическое значение, так как он позволяет рассчитывать тепловые эффекты многих химических реакций, которые не могут быть измерены экспериментально, или это измерение сопряжено с большими трудностями. Особенно удобно проводить такие расчеты, пользуясь следующими следствиями из закона Гесса.  

\begin{quote}\slshape\noindent
Следствие 1. Тепловой эффект реакции равен сумме энтальпий образования продуктов за вычетом суммы энтальпий образования исходных веществ с учетом стехиометрических коэффициентов \cref{fig:a43184}.  
\end{quote}

\begin{equation*}
\Delta H\  = \ \sum_{}^{}{ni{\Delta}_{f}Hi}\  - \ \sum_{}^{}{nj{\Delta}_{f}Hj},
\end{equation*}

где $\Delta$H — изменение энтальпии соответствующей реакции, $\Delta$$_{f}$Hi и $\Delta$$_{f}$Hj — энтальпии образования продуктов реакции и исходных веществ соответственно.  

\begin{figure}[H] 
\centering
\includegraphics[width = \textwidth, keepaspectratio]{./images/image2v.png}
\caption{Схематическая иллюстрация 1 следствия закона Гесса.}
\label{fig:a43184}
\end{figure}

\begin{quote}\slshape\noindent
Следствие 2. Тепловой эффект реакции равен сумме энтальпий сгорания исходных веществ за вычетом суммы энтальпий сгорания продуктов с учетом стехиометрических коэффициентов.  
\end{quote}

\begin{equation*}
{\Delta}_{r}H\  = \ \sum_{}^{}{n_i{\Delta}_{f}H_i}\  - \ \sum_{}^{}{n_j{\Delta}_{f}H_j},
\end{equation*}

где $\Delta$H — изменение энтальпии соответствующей реакции, $\Delta$$_{c}$Hi и $\Delta$$_{c}$Hj — энтальпии сгорания исходных веществ и продуктов реакции соответственно.  

Это следствие обычно используют для расчета тепловых эффектов реакций органических соединений. Иногда в качестве следствия закона Гесса рассматривают еще одно утверждение.  

\begin{quote}\slshape\noindent
Следствие 3. Тепловой эффект реакции равен сумме энергий разрываемых химических связей за вычетом суммы энергий образующихся связей.  
\end{quote}

Энергией связи A-B называют энергию, необходимую для разрыва связи и разведения образующихся частиц на бесконечное расстояние:  

\begin{equation*}
 AB_{(\text{г})} \rightarrow A_{(\text{г})} + B_{(\text{г})}. 
\end{equation*}

\begin{equation*}
{\Delta}_{r}H\  = \ \sum_{}^{}{ni {E}_{ {\text{с}\text{в}\ (\text{р}\text{а}\text{з}\text{р}\text{ы}\text{в})}}}\  - \ \sum_{}^{}{nj {E}_{ {\text{с}\text{в}\ (\text{о}\text{б}\text{р})}}}.
\end{equation*}

\begin{figure}[H] 
\centering
\includegraphics[width = \textwidth, keepaspectratio]{./images/image3v.png}
\caption{Схематическая иллюстрация 3 следствия закона Гесса.}

\end{figure}

Важно: энергия связи всегда положительна.  

Непосредственно из закона Гесса следует, что тепловые эффекты двух взаимно обратных реакций равны по модулю и обратны по знаку:  

\begin{equation*}
{\Delta}\text{Н}(\text{А} \rightarrow \text{В}) = - {\Delta}\text{Н}(\text{В} \rightarrow \text{А})
\end{equation*}

Чтобы нагляднее понять суть закона Гесса, рассмотрим схему:  

\begin{figure}[H] 
\centering
\includegraphics[width = \textwidth, keepaspectratio]{./images/image4v.png}


\end{figure}

Из которой следует, что:  

\begin{equation*}
{\Delta}\text{Н}_{1} + {\Delta}\text{Н}_{2} + {\Delta}\text{Н}_{3} = {\Delta}\text{Н}_{4} + {\Delta}\text{Н}_{5}.
\end{equation*}

Определим энтальпию представленного ниже процесса, зная энтальпии образования веществ, участвующих в нем:  

\begin{equation*}
CO + H_{2}O \rightarrow CO_{2} + H_{2} + {\Delta}H_{1}.
\end{equation*}

Распишем подробнее реакции образования веществ, из которых мы сможем составить процесс:  

\begin{equation*}
C + 0,5O_{2} \rightarrow CO + {\Delta}H_{2};
\end{equation*}

\begin{equation*}
C + O_{2} \rightarrow CO_{2} + {\Delta}H_{3};
\end{equation*}

\begin{equation*}
H_{2} + 0,5O_{2} \rightarrow H_{2}O + {\Delta}H_{4}.
\end{equation*}

Группируя упомянутые выше реакции, получаем:  

\begin{equation*}
{\Delta}H_{1} = \ {\Delta}H_{3} - {\Delta}H_{2} - {\Delta}H_{4}.
\end{equation*}

Решим другую задачу, определим энтальпию образования угарного газа:  

\begin{equation*}
C + O_{2} \rightarrow CO_{2} - 393,5\ \text{к}\text{Д}\text{ж}/\text{м}\text{о}\text{л}\text{ь};
\end{equation*}

\begin{equation*}
C + 0,5O_{2} \rightarrow CO + {\Delta}H_{1};
\end{equation*}

\begin{equation*}
CO + 0,5O_{2} \rightarrow CO_{2} - 283\ \text{к}\text{Д}\text{ж}/\text{м}\text{о}\text{л}\text{ь}.
\end{equation*}

Сгруппировав приведенные ниже реакции, получаем:  

\begin{equation*}
- 393,5 - {\Delta}H_{1} = - 283.
\end{equation*}

${\Delta}H_{1}$= -110,5 кДж/моль.  

\subsubsection{Цикл Борна-Габера}

Существует способ для расчета энергии образования кристаллической решетки, основанный на законе Гесса. Это цикл Борна-Габера, который мы рассмотрим более подробно на примере реакции Na (тв) + 0,5Cl$_{2}$ (г) = NaCl (тв). Представим его на схеме - \cref{fig:2d9ec8}.  

Рассмотрим все представленные на схеме реакции детальнее:  

Для начала натрий переводится из твердого состояния в газовую фазу, минуя жидкое. Данному процессу соответствует энтальпия сублимации:  

\begin{equation*}
{Na}_{(\text{т}\text{в})} \rightarrow {Na}_{(\text{г})} + {\Delta}H_{\text{с}\text{у}\text{б}\text{л}}
\end{equation*}

\begin{figure}[H] 
\centering
\includegraphics[width = \textwidth, keepaspectratio]{./images/image5v.png}
\caption{Схематическая иллюстрация примера цикла Борна-Габера.}
\label{fig:2d9ec8}
\end{figure}

Далее натрий отдает свой электрон, становясь ионом. Реакция проходит с выделением энтальпии ионизации:  

\begin{equation*}
{Na}_{(\text{г})}^{0} \rightarrow {Na}_{(\text{г})}^{+} + e^{-} + {\Delta}H_{\text{и}\text{о}\text{н}\text{и}\text{з}\text{а}\text{ц}}
\end{equation*}

Переходим к хлору. Молекула хлора диссоциирует на атомы, реакцию сопровождает энтальпия диссоциации хлора на атомы:  

\begin{equation*}
{Cl}_{2(\text{г})} \rightarrow 2{Cl}_{(\text{г})}^{.} + {\Delta}H_{\text{д}\text{и}\text{с}}
\end{equation*}

Затем атом хлора забирает на себя дополнительный электрон, переходя в ион. Данную реакцию характеризует энергия сродства к электрону:  

\begin{equation*}
{Cl}_{(\text{г})}^{.} + e^{-} \rightarrow {Cl}_{(\text{г})}^{-} + A_{+ e^{-}}
\end{equation*}

Далее ионы натрия и хлора, взаимодействуя между собой, образуют кристаллическую решетку хлорида натрия. Данному процессу соответствует энергия образования кристаллической решетки:  

\begin{equation*}
{Na}_{(\text{г})}^{+} + {Cl}_{(\text{г})}^{-} \rightarrow {NaCl}_{(\text{т}\text{в})} + E_{\text{к}\text{р}\text{и}\text{с}\text{т}}
\end{equation*}

Запишем еще одну реакцию, необходимую для создания полной картины. Ее характеризует энтальпия образования хлорида натрия:  

\begin{equation*}
{Na}_{(\text{т}\text{в})} + 0,5{Cl}_{2\ (\text{г})} \rightarrow {NaCl}_{(\text{т}\text{в})} + {\Delta}_{f}^{{^\circ}}{H(NaCl)}
\end{equation*}

Приравнивая в соответствии с законом Гесса энтальпии и энергии всех упомянутых выше реакций, получаем выражение для расчёта энтальпии ${\Delta}_{f}^{{^\circ}}{H(NaCl)}$:  

\begin{equation*}
{\Delta}_{f}^{{^\circ}}{H(NaCl)} = {\Delta}H_{\text{с}\text{у}\text{б}\text{л}} + {\Delta}H_{\text{и}\text{о}\text{н}\text{и}\text{з}\text{а}\text{ц}} + 0,5{\Delta}H_{\text{д}\text{и}\text{с}} + A_{+ e^{-}} + E_{\text{к}\text{р}\text{и}\text{с}\text{т}}
\end{equation*}

При этом важно отметить, что экспериментально энергию кристаллической решетки нельзя определить, просто измерив тепловой эффект реакции взаимодействия твердого натрия и газообразного хлора. Для этого необходимо провести серию экспериментов и измерить тепловой эффект в каждом из них согласно схеме, представленной выше. Аналогичные эксперименты были проведены для большого числа соединений, кристаллическая решетка которых считается ионной.  

Резюмируя, запишем формулу для расчета энтальпии образования кристаллической решетки в общем виде:  

\begin{equation*}
A_{(\text{т}\text{в})} + B_{(\text{г})} \rightarrow {AB}_{(\text{т}\text{в})}
\end{equation*}

\begin{equation*}
{\Delta}_{f}^{{^\circ}}{H(AB)} = {\Delta}H_{\text{с}\text{у}\text{б}\text{л}(A)} + {\Delta}H_{\text{и}\text{о}\text{н}\text{и}\text{з}\text{а}\text{ц}(A)} + {\Delta}H_{\text{д}\text{и}\text{с}(B - B)} + A_{+ e^{-}(B)} + E_{\text{к}\text{р}\text{и}\text{с}\text{т}(AB)}
\end{equation*}

Представим все описанные процессы и соответствующие им энергии в виде таблицы:  

\begingroup
\centering
\begin{longtblr}[ caption={Типы энтальпий различных процессов}]{colspec={Q[c]Q[c]Q[c]}, hlines,vlines}
 Вид превращения          &  Процесс                                             &  Обозначение                        \\
 Плавление                &  $\text{т}\text{в} \rightarrow \text{ж}$                                &  ${\Delta}_{fus}H$                \\
 Испарение                &  $\text{ж} \rightarrow \text{г}$                                 &  ${\Delta}_{vap}H$                \\
 Сублимация (возгонка)    &  $\text{т}\text{в} \rightarrow \text{г}$                                &  ${\Delta}_{sub}H$                \\
 Растворение              &  $\text{р}\text{а}\text{с}\text{т}\text{в}\text{о}\text{р}.\ \text{в}\text{е}\text{щ} - \text{в}\text{о} \rightarrow \text{р}\text{а}\text{с}\text{т}\text{в}\text{о}\text{р}$          &  ${\Delta}_{sol}H$                \\
 Гидратация               &  $X_{\text{г}}^{\pm} \rightarrow X_{\text{в}\text{о}\text{д}\text{н}}^{\pm}$          &  ${\Delta}_{hyd}H$                \\
 Атомизация               &  $\text{ч}\text{а}\text{с}\text{т}\text{и}\text{ц}\text{ы}\ (\text{т}\text{в},\ \text{ж},\ \text{г}) \rightarrow \text{а}\text{т}\text{о}\text{м}\text{ы}\ (\text{г}\text{а}\text{з})$  &  ${\Delta}_{at}H$                 \\
 Ионизация                &  $X_{\text{г}} \rightarrow X_{\text{г}}^{+} + e_{\text{г}}^{-}$         &  ${\Delta}_{ion}H$                \\
 Присоединение электрона  &  $X_{\text{г}} + e_{\text{г}}^{-} \rightarrow X_{\text{г}}^{-}$         &  ${\Delta}_{eg}H\ (A_{+ e^{-}})$  \\

\end{longtblr}
\endgroup

\subsubsection{Закон Кирхгофа}

Важно понимать, что энтальпия является функцией от температуры, иными словами зависит от нее. В связи с этим существует формула, позволяющая производить перерасчет от одной температуры к другой (закон Кирхгофа). Однако, прежде чем ввести данную формулу, необходимо условиться о понятии теплоемкости.  

Теплоемкость (C) - это экспериментально измеряемая экстенсивная величина. Мольная теплоемкость — это количество теплоты, необходимое для нагревания 1 моль вещества на 1 градус. Обычно используют величину теплоемкости при постоянном давлении или объеме:  

\begin{equation*}
C_{p} = \left( \frac{\partial H}{\partial T} \right)_{p}\ \ \ \ (7);
\end{equation*}

\begin{equation*}
C_{V} = \left( \frac{\partial U}{\partial T} \right)_{V}\ \ \ \ (8).
\end{equation*}

Теплоемкость идеального газа от температуры не зависит, однако для всех остальных реальных объектов данная зависимость присутствует и выражается, например, через полиномиальные интерполяционные формулы, которые позволяют корректно восстановить значение теплоемкости при любой температуре только внутри указанного температурного интервала. К таким выражениям относятся часто используемые уравнения:  

\begin{equation*}
C_{p} = a + bT\ \ \ \ \ (9);
\end{equation*}

\begin{equation*}
C_{p} = a + bT\  + cT^{- 2}\ \ \ \ \ (10),
\end{equation*}

где a, b и c - это некоторые эмпирические константы.  

Как и в любом другом случае для определения изменения некоторой функции от той или иной переменной необходимо рассмотреть производную. В данном случае это будет:  

\begin{equation*}
\frac{{d{\Delta}}_{r}H_{T}^{0}}{dT} = {\Delta}_{r}C_{p}\ \ \ \ \ \ (11).
\end{equation*}

Таким образом нами был записан закон Кирхгофа в дифференциальной форме.  

Если сделать допущение об отсутствии зависимости теплоемкости от температуры в данном температурном диапазоне и проинтегрировать выражение (11) от Т1 до Т2, то можно получить приращение энтальпии, которое также принято называть высокотемпературным составляющим энтальпии:  

\begin{equation*}
{\Delta}H = H(T_{2}) - H(T_{1}) = \int_{T_{1}}^{T_{2}}{C_{p}dT = C_{p}(T_{2} - T_{1})}\ \ \ \ \ (12).
\end{equation*}

Выполнив незамысловатое преобразование уравнения (11), получаем закон Кирхгофа в интегральной форме (13):  

\begin{equation*}
{\Delta}_{r}H(T_{2}) = {\Delta}_{r}H(T_{1}) + \ {{\Delta}_{r}C}_{p}(T_{2} - T_{1})\ \ \ \ \ (13);
\end{equation*}

\begin{equation*}
{{\Delta}_{r}C}_{p} = \sum_{i}^{}{\alpha_{i}C_{p\ (\text{п}\text{р}\text{о}\text{д})}} - \sum_{i}^{}{\alpha_{i}C_{p\ (\text{р}\text{е}\text{а}\text{г})}\ \ \ \ \ (14),}
\end{equation*}

где $\alpha_{i}$ - соответствующий стехиометрический коэффициент перед i-м веществом.  

\ul{Уровень I}  

\ul{Задача 1}  

При сгорании 1 моль твердой серы в токе кислорода выделяется 297 кДж/моль теплоты. Определите массу серы, которую сжег преподаватель, демонстрируя ученикам голубое пламя, если в результате выделилось 92,07 кДж теплоты.  

\ul{Ответ:} 9,92 г.  

\ul{Задача 2}  

Стандартная теплота сгорания кристаллического бора равна 1264 кДж/моль. Юному химику стало интересно, хватит ли теплоты от сгорания 1 г кристаллического бора для того, чтобы нагреть 250 мл воды в стакане, если известно, что для этого понадобится примерно 90 кДж тепла. Помогите юному химику найти ответ на интересующий его вопрос.  

\ul{Ответ:} 113,76 кДж. Да, хватит.  

\ul{Задача 3}  

Однажды юному алхимику Аль-Бируни принесли на анализ порошок - смесь меди и цинка массой 10 г. Аль-Бируни, недолго думая, полностью растворил ее в соляной кислоте и измерил выделившеюся в результате реакции теплоту, которая составила 10,19 кДж. Помогите юному алхимику определить массы меди и цинка в смеси.  

Дополнительно известно, что теплота реакции растворения 1 моль цинка в соляной кислоте составляет 165,7 кДж/моль.  

\ul{Ответ:} m(Zn) = 4 г, m(Cu) = 6 г.  

\ul{Задача 4 («Ломоносов»)}  

Д.И.Менделеев в учебнике «Основы химии» писал: «$\ldots$ реакции между цинком и слабой (много воды содержащею) серной кислотой развивают на 65 вес. ч. цинка около 38 тыс. кал. тепла, а для 56 вес. ч. железа $\ldots$ отделяется около 25 тыс. кал. тепла (образуется FeSO4 ).». Составьте термохимические уравнения описанных реакций (1 кал $\approx$4.2 Дж) и определите, сколько теплоты выделится при полном растворении 13 г цинка в растворе FeSO4. \ul{Ответ:} 11 кДж.  

\ul{Задача 5}  

На первом коллоквиуме по термохимии профессор задал своим студентам несложную задачу: «Определите стандартную энтальпию образования углекислого газа, если известны следующие данные:  

\begin{equation*}
C_{(\text{т}\text{в})} + \frac{1}{2}O_{2\ (\text{г})} = {CO}_{\ (\text{г})},\ {{\Delta}H}_{1}^{o} = - 110\ \text{к}\text{Д}\text{ж}/\text{м}\text{о}\text{л}\text{ь}
\end{equation*}

${CO}_{(\text{г})} + \frac{1}{2}O_{2\ (\text{г})} = {CO}_{2\ (\text{г})},\ {{\Delta}H}_{2}^{o} = - 284\ \text{к}\text{Д}\text{ж}/\text{м}\text{о}\text{л}\text{ь}$»  

\ul{Ответ:} -384 кДж/моль.  

\ul{Задача 6}  

Широко распространена реакция с участием кислотного оксида Х ($\omega$(O) = 49,95%) и кислорода, проводимая в присутствии катализатора V$_{2}$O$_{5}$ при температуре 673К и давлении 1 атм, где в качестве единственного продукта образуется оксид Y. Рассчитайте объем вещества Х, взятого для данного превращения, если в условиях реакции выделилось 296,7 кДж теплоты. Выход реакции 70%.  

Справочные данные: $\Delta$$_{f}$Н$^{\circ}$$_{673}$(Х $_{(\text{г}\text{а}\text{з})}$) = -296,90 кДж/моль, $\Delta$$_{f}$Н$^{\circ}$$_{673}$(Y $_{(\text{г}\text{а}\text{з})}$) = -395,8 кДж/моль.  

\ul{Ответ:} 165,66 л  

\ul{Задача 7}  

Изопентан (C$_{5}$H$_{12}$) при обычных условиях представляет из себя легковоспламеняющуюся жидкость. При нагревании выше 30$^{\circ}$С он переходит в газовую фазу. Определите энтальпию реакции испарения изопентана, если известно, что энтальпия образования C$_{5}$H$_{12 (\text{г})}$ составляет: -154,47 кДж/моль, а для C$_{5}$H$_{12 (\text{ж})}$: -179,28 кДж/моль.  

\ul{Ответ:} 24,81 кДж/моль.  

\ul{Задача 8}  

При фторировании простого вещества Х в электрическом разряде образуется бесцветный невоспламеняющийся токсичный газ А ($\omega$(Х)=19,718%). Известно, что стандартная энтальпия образования А составляет -132 кДж/моль, а данные по энергиям связи для Х и F$_{2}$ приведены в таблице:  

\begingroup
\centering
\begin{longtblr}[ ]{colspec={Q[c]Q[c]}, hlines,vlines}
 Х              &  945 кДж/моль  \\
 F$_{2}$  &  159 кДж/моль  \\

\end{longtblr}
\endgroup

Определите вещества Х и А. Вычислите энергию связи X-F в А.  

\ul{Ответ:} Х - N$_{2}$ и A - F$_{2}$; 281 кДж/моль.  

\ul{Задача 9}  

Человек получает энергию в результате процесса расщепления глюкозы в его организме. Суммарно реакция выглядит так:  

\begin{equation*}
C_{6}H_{12}O_{6} + 6O_{2} = \ 6CO_{2} + 6H_{2}O + 38\text{А}\text{Т}\text{Ф}
\end{equation*}

<table> <colgroup> <col style="width: 24%" /> <col style="width: 25%" /> <col style="width: 25%" /> <col style="width: 25%" /> </colgroup> <thead> <tr class="header"> <th>Возраст, лет</th> <th>Физическая активность</th> <th>Женщины, ккал/сутки</th> <th>Мужчины, ккал/сутки</th> </tr> </thead> <tbody> <tr class="odd"> <td>18-40</td> <td><table> <colgroup> <col style="width: 100%" /> </colgroup> <thead> <tr class="header"> <th>Низкая</th> </tr> </thead> <tbody> <tr class="odd"> <td>Средняя</td> </tr> <tr class="even"> <td>Высокая</td> </tr> </tbody> </table></td> <td><table> <colgroup> <col style="width: 100%" /> </colgroup> <thead> <tr class="header"> <th>2400</th> </tr> </thead> <tbody> <tr class="odd"> <td>2550</td> </tr> <tr class="even"> <td>2700</td> </tr> </tbody> </table></td> <td><table> <colgroup> <col style="width: 100%" /> </colgroup> <thead> <tr class="header"> <th>2800</th> </tr> </thead> <tbody> <tr class="odd"> <td>3000</td> </tr> <tr class="even"> <td>3200</td> </tr> </tbody> </table></td> </tr> <tr class="even"> <td>40-60</td> <td><table> <colgroup> <col style="width: 100%" /> </colgroup> <thead> <tr class="header"> <th>Низкая</th> </tr> </thead> <tbody> <tr class="odd"> <td>Средняя</td> </tr> <tr class="even"> <td>Высокая</td> </tr> </tbody> </table></td> <td><table> <colgroup> <col style="width: 100%" /> </colgroup> <thead> <tr class="header"> <th>2200</th> </tr> </thead> <tbody> <tr class="odd"> <td>2350</td> </tr> <tr class="even"> <td>2500</td> </tr> </tbody> </table></td> <td><table> <colgroup> <col style="width: 100%" /> </colgroup> <thead> <tr class="header"> <th>2600</th> </tr> </thead> <tbody> <tr class="odd"> <td>2800</td> </tr> <tr class="even"> <td>2900</td> </tr> </tbody> </table></td> </tr> <tr class="odd"> <td>20-40</td> <td>Спортсмены</td> <td>3500</td> <td>4500</td> </tr> <tr class="even"> <td>60-70</td> <td>Средняя</td> <td>2100</td> <td>2350</td> </tr> </tbody> </table>  

Сколько выделяется тепла в результате приведенной выше реакции у спортсменки 38 лет, если она потребляет свою суточную норму калорий? На сколько необходимо уменьшить суточную дозировку калорий мужчине 56 лет с низкой физической активностью, если за день у него распадается 900г глюкозы?  

Справочные данные: $\Delta$$_{f}$Н$^{\circ}$$_{298}$(CO$_{2}$) = -393,51 кДж/моль, $\Delta$$_{f}$Н$^{\circ}$$_{298}$(C$_{6}$H$_{12}$O$_{6}$) = -1273,3 кДж/моль, $\Delta$$_{f}$Н$^{\circ}$$_{298}$(H$_{2}$O) = -285,83 кДж/моль; 1 ккал = 4,184 кДж  

\ul{Ответ:} 14,644МДж, 573 ккал  

\ul{Задача 10}  

\begin{quote}\slshape\noindent
 Даны следующие реакции:  
\end{quote}

\begin{equation*}
1/2{Cl}_{2} + {1/2F}_{2} = ClF\ \ \ \ \ \ (1)
\end{equation*}

\begin{equation*}
{1/2Cl}_{2} + 3/2F_{2} = ClF_{3}\ \ \ \ \ \ (2)
\end{equation*}

Известно, что Е(Cl-Cl) = 243 кДж/моль, Е(F-F) = 159 кДж/моль, $\Delta$$_{f}$Н$^{\circ}$$_{298}$(ClF) = -50,3 кДж/моль, $\Delta$$_{f}$Н$^{\circ}$$_{298}$(ClF$_{3}$) = -158,9 кДж/моль. Определите энергию связей в молекулах ClF и ClF$_{3}$ и объясните причину их различия.  

\ul{Ответ:} 251,3 кДж/моль, 173 кДж/моль  

\ul{Уровень II}  

\ul{Задача 11}  

Рассчитайте энтальпию образования сульфата цинка из простых веществ при Т=298К на основании следующих данных:  

\begin{equation*}
ZnS_{\ (\text{т}\text{в})} \rightarrow {Zn}_{\ (\text{т}\text{в})} + S_{\ (\text{т}\text{в})},\ \ {\Delta}H_{1} = 200,5\ \text{к}\text{Д}\text{ж}/\text{м}\text{о}\text{л}\text{ь}
\end{equation*}

\begin{equation*}
2ZnS_{\ (\text{т}\text{в})} + 3O_{2\ (\text{г})} \rightarrow 2ZnO_{\ (\text{т}\text{в})} + 2SO_{2\ (\text{г})},\ \ {\Delta}H_{2} = - 893,5\ \text{к}\text{Д}\text{ж}/\text{м}\text{о}\text{л}\text{ь}
\end{equation*}

\begin{equation*}
2SO_{2\ (\text{г})} + O_{2\ (\text{г})} \rightarrow 2SO_{3\ (\text{ж})},\ \ {\Delta}H_{3} = - 198,2\ \text{к}\text{Д}\text{ж}/\text{м}\text{о}\text{л}\text{ь}
\end{equation*}

\begin{equation*}
ZnSO_{4\ (\text{т}\text{в})} \rightarrow ZnO_{\ (\text{т}\text{в})} + SO_{3\ (\text{ж})},\ \ {\Delta}H_{4} = 235,0\ \text{к}\text{Д}\text{ж}/\text{м}\text{о}\text{л}\text{ь}
\end{equation*}

\ul{Ответ:} -981,35 кДж/моль  

\ul{Задача 12}  

Для реакции сгорания 1 моль метана в токе кислорода рассчитайте энергию связи (С-Н) в СН$_{4}$, если дополнительно известно, что:  

$\Delta$$_{f}$H(СН$_{4}$) = -74,8 кДж/моль  

$\Delta$$_{f}$H(Н$_{2}$О) =-285,8 кДж/моль  

$\Delta$$_{f}$H(СО$_{2}$) = -393,5 кДж/моль  

Е(С=О) = 800 кДж/моль  

Е(О-Н) = 495 кДж/моль  

Е(О=О) = 498 кДж/моль  

\ul{Ответ:} 423,4 кДж/моль.  

\ul{Задача 13}  

При образовании 30 л метана из графита и водорода (при 25$^{o}$С и нормальном давлении) выделилось 92 кДж теплоты. Средняя энергия связи С-Н в метане равна 413 кДж/моль, энергия связи Н-Н составляет 436 кДж/моль. Рассчитайте теплоту испарения графита (в кДж/моль).  

\ul{Ответ:} 705 кДж/моль.  

\ul{Задача 14 («Ломоносов» )}  

При полном сгорании 134,4 л (н.у.) смеси метана и пропана, содержащей 75% метана по объему, выделилось 7336 кДж теплоты. Рассчитайте теплоту образования пропана, если теплоты образования метана, CO$_{2}$ и Н$_{2}$O равны 74,81, 393,5 и 285,8 кДж/моль соответственно.  

\ul{Ответ:} 103,9 кДж/моль.  

\ul{Задача 15}  

Найдите $\Delta$$_{r}$H$_{298}$$^{o}$ для реакции:  

\begin{equation*}
CH_{4} + Cl_{2} \rightarrow CH_{3}Cl + HCl
\end{equation*}

Если энтальпии сгорания для метана, хлорметана и водорода равны соответственно -890,6, -689,8 и -285,8 кДж/моль, а энтальпия образования HCl равна -92,3 кДж/моль.  

\ul{Ответ:} -99,6 кДж/моль.  

\ul{Задача 16}  

Используя представленные данные, рассчитайте $\Delta$$_{f}$H$^{o}$ для SOCl$_{2}$.  

\begin{equation*}
2P_{(\text{т}\text{в})} + 3{Cl}_{2\ (\text{г})} \rightarrow {2PCl}_{3\ (\text{ж})},\ \ {\Delta}_{r}H_{1}^{{^\circ}} = - 642,0\ \text{к}\text{Д}\text{ж}/\text{м}\text{о}\text{л}\text{ь}\ 
\end{equation*}

\begin{equation*}
{PCl}_{3\ (\text{ж})} + {Cl}_{2\ (\text{г})} \rightarrow {PCl}_{5\ (\text{т}\text{в})},\ \ {\Delta}_{r}H_{2}^{{^\circ}} = - 125,0\ \text{к}\text{Д}\text{ж}/\text{м}\text{о}\text{л}\text{ь}\ 
\end{equation*}

\begin{equation*}
6PCl_{5\ (\text{т}\text{в})} + P_{4}O_{10\ (\text{т}\text{в})} \rightarrow {10POCl}_{3\ (\text{ж})},\ \ {\Delta}_{r}H_{3}^{{^\circ}} = - 308,0\ \text{к}\text{Д}\text{ж}/\text{м}\text{о}\text{л}\text{ь}\ 
\end{equation*}

\begin{equation*}
4P_{(\text{т}\text{в})} + {5O}_{2\ (\text{г})} \rightarrow P_{4}O_{10\ (\text{т}\text{в})},\ \ {\Delta}_{r}H_{4}^{{^\circ}} = - 2988,0\ \text{к}\text{Д}\text{ж}/\text{м}\text{о}\text{л}\text{ь}\ 
\end{equation*}

\begin{equation*}
S_{(\text{т}\text{в})} + O_{2\ (\text{г})} \rightarrow {SO}_{2\ (\text{г})},\ \ {\Delta}_{r}H_{5}^{{^\circ}} = - 296,9\ \text{к}\text{Д}\text{ж}/\text{м}\text{о}\text{л}\text{ь}\ 
\end{equation*}

\begin{equation*}
{PCl}_{5\ (\text{т}\text{в})} + {SO}_{2\ (\text{г})} \rightarrow {POCl}_{3\ (\text{ж})} + {SOCl}_{2\ (\text{ж})},\ \ {\Delta}_{r}H_{6}^{{^\circ}} = - 101,1\ \text{к}\text{Д}\text{ж}/\text{м}\text{о}\text{л}\text{ь}\ 
\end{equation*}

\ul{Ответ:} -246,8 кДж/моль.  

\ul{Задача 17}  

Найдите энергию кристаллической решетки BaCl$_{2}$. Если известны следующие данные:  

$\Delta$$_{f}$H(BaCl$_{2}$) = -859,418 кДж/моль,  

$\Delta$$_{\text{а}\text{т}}$H(Ba) = 192,28 кДж/моль,  

$\Delta$$_{\text{и}\text{о}\text{н}}$H(Ba$\rightarrow$Ba$^{2+}$) = 1462,16 кДж/моль,  

$\Delta$$_{\text{д}\text{и}\text{с}\text{с}}$H(Cl$_{2}$) = 238,26 кДж/моль,  

$\Delta$$_{r}$H(Cl$\rightarrow$Cl$^{-}$) = -363,66 кДж/моль,  

\ul{Ответ:} = -2388,458 кДж/моль.  

\ul{Задача 18}  

Измерить теплоту сгорания углерода до CO в ограниченном количестве кислорода практически невозможно, потому что продукт реакции всегда будет состоять из смеси CO и CO2. Однако теплоту полного сгорания углерода до CO2 в избытке кислорода измерить можно:  

\begin{equation*}
C_{(\text{г}\text{р}\text{а}\text{ф}\text{и}\text{т})} + \ O_{2(\text{г})} \rightarrow CO_{2(\text{г})},\Delta cH_{1} = - 94,0518\ \text{к}\text{к}\text{а}\text{л}/\text{м}\text{о}\text{л}\text{ь}
\end{equation*}

Можно измерить и теплоту сгорания CO до CO2:  

\begin{equation*}
CO_{(\text{г})} + 0,5O_{2(\text{г})} \rightarrow CO_{2(\text{г})},\Delta cH_{2} = - 67,6361\ \text{к}\text{к}\text{а}\text{л}/\text{м}\text{о}\text{л}\text{ь}
\end{equation*}

На основании этих данных рассчитайте теплоту образования CO и приведите ответ в кДж/моль, зная, что 1 кал = 4,184 Дж.  

\ul{Ответ:} -110,52 кДж/моль  

\ul{Задача 19}  

Рассчитайте энтальпию образования газообразного иодоводорода, используя данные о тепловых эффектах следующих реакций:  

\begin{equation*}
H_{2} + Cl_{2} \rightarrow 2HCl_{(\text{г})},\ {\Delta}H_{1} = - 44,12\ \text{к}\text{к}\text{а}\text{л}/\text{м}\text{о}\text{л}\text{ь}
\end{equation*}

\begin{equation*}
HCl_{(\text{г})} + aq \rightarrow HCl_{(aq)},\ {\Delta}H_{2} = - 17,96\ \text{к}\text{к}\text{а}\text{л}/\text{м}\text{о}\text{л}\text{ь}
\end{equation*}

\begin{equation*}
HI_{(\text{г})} + aq \rightarrow HI_{(aq)},\ {\Delta}H_{3} = - 19,21\ \text{к}\text{к}\text{а}\text{л}/\text{м}\text{о}\text{л}\text{ь}
\end{equation*}

\begin{equation*}
KOH_{(aq)} + HCl_{(aq)} \rightarrow KCl_{(aq)},\ {\Delta}H_{4} = - 13,74\ \text{к}\text{к}\text{а}\text{л}/\text{м}\text{о}\text{л}\text{ь}
\end{equation*}

\begin{equation*}
KOH_{(aq)} + HI_{(aq)} \rightarrow KI_{(aq)},\ {\Delta}H_{5} = - 13,67\ \text{к}\text{к}\text{а}\text{л}/\text{м}\text{о}\text{л}\text{ь}
\end{equation*}

\begin{equation*}
Cl_{2(\text{г})} + 2KI_{(aq)} \rightarrow 2KCl_{(aq)} + I_{2(\text{т}\text{в})},\ {\Delta}H_{6} = - 52,42\ \text{к}\text{к}\text{а}\text{л}/\text{м}\text{о}\text{л}\text{ь}\ 
\end{equation*}

\ul{Ответ:} 5,33 ккал/моль  

\ul{Задача 20 (ВсОШ ЗЭ 2022-2023)}  

Известно, что фосфор в реакции с хлором способен образовывать два хлорида, которые являются газообразными при температуре реакции. Известны энтальпии этих процессов:  

\begin{equation*}
(1)\ 2P_{(\text{т}\text{в})} + 3Cl_{2} \rightarrow 2PCl_{3(\text{г})},\ \ \Delta H_{1} = - 581\ \text{к}\text{Д}\text{ж}/\text{м}\text{о}\text{л}\text{ь}\ 
\end{equation*}

\begin{equation*}
(2)\ 2P_{(\text{т}\text{в})} + 5Cl_{2} \rightarrow 2PCl_{5(\text{г})},\ \ \Delta H_{2} = - 751\ \text{к}\text{Д}\text{ж}/\text{м}\text{о}\text{л}\text{ь}
\end{equation*}

Если после реакции не конденсировать хлориды фосфора, а поддерживать высокую температуру, между газообразными хлоридами установится равновесие в соответствии с реакцией:  

\begin{equation*}
(3)\ PCl_{3(\text{г})} + Cl_{2} \rightarrow PCl_{5(\text{г})},\ \ \Delta H_{3} = \ ?\ \text{к}\text{Д}\text{ж}/\text{м}\text{о}\text{л}\text{ь}
\end{equation*}

\begin{enumerate}\keepwithnext
    \setcounter{enumi}{0}
    \item Вычислите энтальпию реакции 3.  
\end{enumerate}

Известно, что энергии связи P-Cl в двух хлоридах неравноценны: в PCl$_{3}$ энергия связи на 23,53 % выше, чем в PCl$_{5}$. Известно также, что энергия связи в молекуле хлора составляет 236 кДж/моль.  

\begin{enumerate}\keepwithnext
    \setcounter{enumi}{1}
    \item Установите энергии связи P-Cl в PCl$_{3}$ и PCl$_{5}$.  
\end{enumerate}

\ul{Ответ:} -85 кДж/моль, P-Cl в PCl$_{3}$ - 306,2 кДж/моль, P-Cl в PCl$_{5}$ - 247,9 кДж/моль  

\ul{Задача 21}  

Рассчитайте энтальпию образования фреонов C$_{2}$F$_{3}$Cl$_{3}$, CF$_{3}$Cl, CHFCl$_{2}$, C$_{2}$HF$_{5}$ и C$_{2}$F$_{4}$H$_{2}$, если известно, что E(C-C) = 321 кДж/моль, Е(C-F) = 486 кДж/моль, Е(C-Cl) = 324 кДж/моль, Е(С-Н) = 400 кДж/моль.  

\ul{Ответ:} 2751 кДж/моль, 1782 кДж/моль, 1210 кДж/моль, 3151 кДж/моль, 3065 кДж/моль  

\ul{Задача 22}  

Рассчитайте количество теплоты, которое выделится при сгорании 4 л (н.у.) сероводорода. $\Delta$$_{f}$Н$^{\circ}$$_{298}$(SO$_{2}$) = -296,9 кДж/моль, $\Delta$$_{f}$Н$^{\circ}$$_{298}$(H$_{2}$O) = -285,8 кДж/моль, $\Delta$$_{\text{д}\text{и}\text{с}}$Н$^{\circ}$$_{298}$(H$_{2}$) = 432,0 кДж/моль, $\Delta$$_{\text{а}\text{т}}$Н$^{\circ}$$_{298}$(S) = 273,0 кДж/моль, E(S-H) = 363,1 кДж/моль.  

\ul{Ответ:} 101,1 кДж/моль  

\ul{Задача 23}  

Вычислите энергию кристаллической решетки оксида кальция: если известно, что стандартная энтальпия реакции  

\begin{equation*}
Ca{(HCOO)}_{2} \rightarrow HCOH + CaO + {CO}_{2}
\end{equation*}

составляет $\Delta$$_{r}$Н$^{\circ}$ = 240,9 кДж/моль. При расчетах используйте следующие данные: A(O$\rightarrow$O$^{2-}$) = 710,0 кДж/моль, I(Ca$\rightarrow$Ca$^{2+}$) = 1735,2 кДж/моль  

\begingroup
\centering
\begin{longtblr}[ ]{colspec={Q[c]Q[c]Q[c]Q[c]Q[c]Q[c]}, hlines,vlines}
                            &  O      &  Ca$_{(\text{г})}$  &  СO$_{2}$  &  HCOH    &  Ca(HCOO)$_{2}$  \\
 $\Delta$$_{f}$Н$^{\circ}$, кДж/моль  &  246,8  &  178,0             &  -393,5          &  -115,8  &  -1385,3               \\

\end{longtblr}
\endgroup

\ul{Ответ:} -3504,9 кДж/моль  

\ul{Задача 24 (КФУ 2023)}  

Важность данных молекул для атмосферной химии связана с тем, что при их распаде на свету образуются ОН-радикалы, участвующие в дальнейшем в других процессах. Ситуацию осложняет существование цис- и транс-изомеров HONO. Ниже представлены термохимические уравнения некоторых процессов, происходящих в атмосфере с их участием.  

\begin{equation*}
(1)\ H + NO_{2} \rightarrow \text{ц}\text{и}\text{с} - HONO,\ \ \Delta_{r}H_{1} = - 327,8\ \text{к}\text{Д}\text{ж}/\text{м}\text{о}\text{л}\text{ь}\ 
\end{equation*}

\begin{equation*}
(2)\ H + NO \rightarrow HNO,\ \ \Delta_{r}H_{2} = - 201,9\ \text{к}\text{Д}\text{ж}/\text{м}\text{о}\text{л}\text{ь}\ 
\end{equation*}

\begin{equation*}
(3)\ HNO + H_{2}O \rightarrow HNO_{2} + H_{2},\ \ \Delta_{r}H_{3} = 89,1\ \text{к}\text{Д}\text{ж}/\text{м}\text{о}\text{л}\text{ь}\ 
\end{equation*}

\begin{equation*}
(4)\ H_{2}O \rightarrow H + OH,\ \ \Delta_{r}H_{4} = 498,8\ \text{к}\text{Д}\text{ж}/\text{м}\text{о}\text{л}\text{ь}\ 
\end{equation*}

\begin{equation*}
(5)\ \text{т}\text{р}\text{а}\text{н}\text{с} - HONO \rightarrow OH + NO,\ \ \Delta_{r}H_{5} = 205,8\ \text{к}\text{Д}\text{ж}/\text{м}\text{о}\text{л}\text{ь}\ 
\end{equation*}

\begin{equation*}
(6)\ NO_{2} + H_{2} \rightarrow H_{2}O + NO,\ \ \Delta_{r}H_{6} = - 183,9\ \text{к}\text{Д}\text{ж}/\text{м}\text{о}\text{л}\text{ь}\ 
\end{equation*}

Дополнительно известно, что энергия связи в молекуле водорода равна 436 кДж/моль.  

Рассчитайте энтальпии реакций распада цис-HONO и HNO$_{2}$ на ОН-радикалы и NO, а также энтальпии изомеризации транс-HONO в цис-HONO и транс-HONO в HNO$_{2}$. Сделайте вывод о том, какой из трёх изомеров наиболее стабилен.  

\ul{Ответ:} 206,7 кДж/моль, 175,6 кДж/моль, -0,9 кДж/моль, 30,2 кДж/моль  

\ul{Задача 25 (МОШ ЗЭ 2023-2024)}  

\begin{enumerate}\keepwithnext
    \setcounter{enumi}{0}
    \item Запишите уравнение реакции полного сгорания метанола (CH$_{3}$OH) и рассчитайте тепловой эффект этой реакции используя теплоту образования веществ.  
    \setcounter{enumi}{1}
    \item Рассчитайте тепловой эффект реакции сгорания метанола, но уже с использованием данных по энергиям связей.  
    \setcounter{enumi}{2}
    \item При условии отсутствия погрешностей при определении энергий связей и теплот образования, как Вы считаете, какой метод расчета теплового эффекта реакции более точный - через E$_{\text{с}\text{в}}$ или через Q$_{f}$?  
    \setcounter{enumi}{3}
    \item Запишите уравнения реакций полного гидролиза хлорида фосфора (V) и оксохлорида фосфора (POCl$_{3}$). Рассчитайте тепловые эффекты этих реакций. На основании полученных данных рассчитайте энергии связей P=O и P-Cl. Примите, что энергия связи не зависит от окружения.  
\end{enumerate}

Справочная информация  

\begingroup
\centering
\begin{longtblr}[ ]{colspec={X[6.0,c]X[11.0,c]X[6.0,c]X[11.0,c]X[5.0,c]X[12.0,c]X[5.0,c]X[12.0,c]}, hlines,vlines}
 Вещ-во            &  Q$_{f}$, кДж/моль  &  Вещ-во                       &  Q$_{f}$, кДж/моль  &  Связь  &  E$_{\text{с}\text{в}}$, кДж/моль  &  Связь          &  E$_{\text{с}\text{в}}$, кДж/моль  \\
 CO$_{2}$    &  393                      &  POCl$_{3}$             &  597                      &  O-H    &  459                       &  O$_{2}$  &  494                       \\
 H$_{2}$O    &  286                      &  H$_{3}$PO$_{4}$  &  1279                     &  C-H    &  410                       &  P-O            &  350                       \\
 CH$_{3}$OH  &  239                      &  HCl                          &  93                       &  C-O    &  358                       &  H-Cl           &  427                       \\
 O$_{2}$     &  0                        &  PCl$_{5}$              &  367                      &  C=O    &  798                       &                 &                            \\

\end{longtblr}
\endgroup

\ul{Ответ:} 1)1452 кДж 2) 1288 кДж 3) 103 кДж, 233 кДж, E(P=O) = 763 кДж/моль, , E(P-Cl) = 284 кДж/моль  

\ul{Уровень III}  

\ul{Задача 26}  

Жуки-бомбардиры из подсемейства жужелиц Brachininae получили название благодаря своеобразному защитному механизму. Они способны прицельно выстреливать 9 из отверстий в задней части брюшка горячую жидкую смесь химических веществ. Температура смеси в момент выстрела достигает 100 $^{\circ}$C, а её выброс сопровождается громким звуком.  

Эти жуки обладают железами внутренней секреции, вырабатывающими смесь гидрохинонов и пероксида водорода. В момент выстрела эти реагенты поступают в реакционную камеру, где смешиваются с раствором природных катализаторов - ферментов. Под действием ферментов происходят химические реакции, в результате которых реакционная смесь разогревается до кипения и выбрасывается наружу через отверстия на кончике брюшка. Этот кончик у Brachininae подвижен и позволяет направлять струю жидкости на врага.  

\begin{enumerate}\keepwithnext
    \setcounter{enumi}{0}
    \item Рассчитайте тепловой эффект реакции окисления гидрохинона до хинона     пероксидом водорода.  
    \setcounter{enumi}{1}
    \item Какая реакция, кроме реакции окисления гидрохинона до хинона, может     служить дополнительным источником теплоты? Ответ подтвердите     расчётом теплового эффекта этой реакции.  
\end{enumerate}

\begin{quote}\slshape\noindent
 Справочная информация  
\begin{figure}[H] 
\centering
\includegraphics[width = \textwidth, keepaspectratio]{/home/yaroslavasus/vzlet_vault/Методичка по физхимии/attachment/image6v.png}


\end{figure}
Теплоты образования воды и пероксида водорода равны 285,8 и 187,  кДж/моль, а теплоты сгорания хинона и гидрохинона до углекислого газ  и воды равны 2746 и 2855 кДж/моль соответственно. Теплоёмкость жидко  воды равна 4,18 Дж/г $\cdot$К  
\end{quote}

\ul{Задача 27 (ВсОШ ЗЭ 2015-2016)}  

Количественной мерой устойчивости кристалла является энергия кристаллической решётки, то есть энергия, которую необходимо затратить, чтобы превратить 1 моль твердого ионного соединения в газ, состоящий из ионов. Например, для хлорида натрия эта энергия равна энтальпии реакции  

\begin{equation*}
NaCl_{(\text{т}\text{в})}\  = \ {Na}_{(\text{г})}^{+}\  + \ {Cl}_{(\text{г})}^{-}.\ 
\end{equation*}

Энергию кристаллической решётки нельзя измерить экспериментально, но можно определить с помощью термохимического цикла, предложенного двумя нобелевскими лауреатами - немецким физиком М. Борном и немецким химиком Ф. Габером.  

Энтальпию образования 1 моль хлорида натрия из твёрдого натрия и газообразного хлора можно измерить:  

\begin{equation*}
Na_{(\text{т}\text{в})}\  + \ \frac{1}{2}Cl_{2\ (\text{г})}\  = \ NaCl_{(\text{т}\text{в})},\ {\Delta}H\  = \  - 411\ \text{к}\text{Д}\text{ж}/\text{м}\text{о}\text{л}\text{ь}.
\end{equation*}

(Положительное значение энтальпии означает, что реакция эндотермическая, а отрицательное - что реакция экзотермическая.)  

Используя цикл Борна-Габера, представим, что этот процесс протекает в пять стадий:  

\begin{enumerate}\keepwithnext
    \setcounter{enumi}{0}
    \item Превращение твёрдого натрия в газ:  
\end{enumerate}

\begin{equation*}
Na_{(\text{т}\text{в})}\  = \ Na_{(\text{г})},\ {\Delta}H_{1}.
\end{equation*}

\begin{enumerate}\keepwithnext
    \setcounter{enumi}{1}
    \item Превращение атомов натрия в положительные ионы:  
\end{enumerate}

\begin{equation*}
Na_{(\text{г})}\  = \ {Na}_{(\text{г})}^{+}\  + \ e^{-},\ {\Delta}H_{2}.
\end{equation*}

\begin{enumerate}\keepwithnext
    \setcounter{enumi}{2}
    \item Диссоциация молекул хлора на атомы:  
\end{enumerate}

\begin{equation*}
Cl_{2\ (\text{г})} = \ 2Cl_{(\text{г})}\ ,\ {\Delta}H_{3}.
\end{equation*}

\begin{enumerate}\keepwithnext
    \setcounter{enumi}{3}
    \item Превращение атомов хлора в отрицательные ионы:  
\end{enumerate}

\begin{equation*}
{Cl}_{(\text{г})} + \ e^{-}\  = {Cl}_{(\text{г})}^{-},\ {\Delta}H_{4}.
\end{equation*}

\begin{enumerate}\keepwithnext
    \setcounter{enumi}{4}
    \item Взаимодействие ионов натрия и хлора с образованием кристаллического     NaCl:  
\end{enumerate}

\begin{equation*}
{Na}_{(\text{г})}^{+} + \ {Cl}_{(\text{г})}^{-} = \ \ NaCl_{(\text{т}\text{в})},\ {\Delta}H_{5}.
\end{equation*}

Задания  

\begin{enumerate}\keepwithnext
    \setcounter{enumi}{0}
    \item Вам приведены значения $\Delta$H перечисленных реакций (в неверном порядке). Расположите указанные значения $\Delta$H реакций в правильном порядке:  
\end{enumerate}

$\Delta$H$_{1}$ = -349 кДж/моль  

$\Delta$H$_{2}$ = 107 кДж/моль  

$\Delta$H$_{3}$ = 244 кДж/моль  

$\Delta$H$_{4}$ = 496 кДж/моль  

Используя приведённые данные, рассчитайте энергию кристаллической решётки хлорида натрия.  

2.Каков знак полученного значения энергии кристаллической решётки (положительный или отрицательный?) Почему? Объясните.  

\begin{enumerate}\keepwithnext
    \setcounter{enumi}{2}
    \item Качественно предскажите, как от энергии кристаллической решётки NaCl будут отличаться энергии решёток LiCl, KСl, NaF и NaBr (больше или меньше). Почему? Объясните.  
    \setcounter{enumi}{3}
    \item Энергия кристаллической решётки хлорида магния равна 2524 кДж/моль. Объясните отличие этой величины от энергии решётки хлорида натрия.  
    \setcounter{enumi}{4}
    \item Известны следующие энтальпии ионизации атомов натрия и магния:  
\end{enumerate}

\begin{equation*}
{Na}_{(\text{г})}^{+} = {Na}_{(\text{г})}^{2 +}\  + \ e^{-},\ {\Delta}H\  = \ 4562\ \text{к}\text{Д}\text{ж}/\text{м}\text{о}\text{л}\text{ь},\ 
\end{equation*}

\begin{equation*}
Mg_{(\text{г})} = {Mg}_{(\text{г})}^{+} + e^{-},\ {\Delta}H\  = \ 738\ \text{к}\text{Д}\text{ж}/\text{м}\text{о}\text{л}\text{ь},\ 
\end{equation*}

\begin{equation*}
{Mg}_{(\text{г})}^{+} = {Mg}_{(\text{г})}^{2 +} + e^{-},{\Delta}H\  = \ 1451\ \text{к}\text{Д}\text{ж}/\text{м}\text{о}\text{л}\text{ь}.\ 
\end{equation*}

Используя эти данные, предложите своё объяснение, почему не существует кристаллических а) NaCl$_{2}$ и б) MgCl.  

\begin{enumerate}\keepwithnext
    \setcounter{enumi}{5}
    \item Энтальпии гидратации ионов Na$^{+}$$_{(\text{г})}$ и Cl$^{-}$$_{(\text{г})}$ равны -406 кДж/моль и -377 кДж/моль соответственно.  
\end{enumerate}

а) Выделяется или поглощается теплота при растворении кристаллического хлорида натрия в воде? Ответ подтвердите расчётом энтальпии реакции растворения.  

б) Сколько теплоты выделится (или поглотится) при растворении в воде 46.8 г кристаллического NaCl? Полезные знания: Энергия электростатического взаимодействия двух зарядов пропорциональна их величинам и обратно пропорциональна расстоянию между ними.  

\ul{Задача 28 (ВсОШ РЭ 2018-2019)}  

При испарении селена в газовой фазе устанавливается сложное равновесиемежду различными циклическими молекулами Se$_{n}$, n = 3, 4, $\ldots$, 8 и Se$_{2}$. Известны энтальпии реакций, которые частично описывают эту систему.  

\begin{equation*}
8{Se}_{(\text{т}\text{в}.)}\  \rightarrow \ Se_{8(\text{г}.)},\ \ \Delta_{r}H_{1}^{{^\circ}} = 40.5\ \text{к}\text{к}\text{а}\text{л}/\text{м}\text{о}\text{л}\text{ь}\ $$ $$3{Se}_{2(\text{г}.)}\  \rightarrow \ {Se}_{6(\text{г}.)},\ \ \Delta_{r}H_{2}^{{^\circ}} = - 71.4\ \text{к}\text{к}\text{а}\text{л}/\text{м}\text{о}\text{л}\text{ь}\ $$ $$2{Se}_{4(\text{г}.)}\  \rightarrow \ {Se}_{8(\text{г}.)},\ \ \Delta_{r}H_{3}^{{^\circ}} = - 35.5\ \text{к}\text{к}\text{а}\text{л}/\text{м}\text{о}\text{л}\text{ь}\ 
\end{equation*}

\begin{equation*}
2{Se}_{2(\text{г}.)}\  \rightarrow \ {Se}_{4(\text{г}.)},\ \ {\ \Delta}_{r}H_{4}^{{^\circ}} = - 31.7\ \text{к}\text{к}\text{а}\text{л}/\text{м}\text{о}\text{л}\text{ь}\ 
\end{equation*}

\begin{equation*}
{Se}_{6(\text{г}.)}\  \rightarrow \ {2Se}_{3(\text{г}.)},\ \ \Delta_{r}H_{5}^{{^\circ}} = 53.4\ \text{к}\text{к}\text{а}\text{л}/\text{м}\text{о}\text{л}\text{ь}\ 
\end{equation*}

\begin{enumerate}\keepwithnext
    \setcounter{enumi}{0}
    \item Определите энтальпии образования Se$_{6}$ и Se$_{3}$ из твердого Se. Объясните с точки зрения строения этих молекул полученное соотношение между энтальпиями образования (какая из них больше и почему?).  
    \setcounter{enumi}{1}
    \item Средняя энергия связи в молекуле Se$_{6}$ равна 49.4 ккал/моль. Определите средние энергии связей в молекуле Se$_{2}$. Объясните соотношение между энергиями связей в Se$_{6}$ и Se$_{2}$.  
\end{enumerate}

Дополнительная информация  

Средняя энергия связи - энергия необходимая для разрыва 1 моль связей данного типа в газообразном веществе  

\ul{Задача 29 (ВсОШ РЭ 2023-2024)}  

Одной из важнейших характеристик топлив и горючих химических веществ является теплота сгорания. При этом для ряда веществ выделяют две теплоты сгорания: высшую и низшую, разница между которыми объясняется разными агрегатными состояниями образующейся воды. Например, для метана высшая теплота сгорания соответствует реакции (1):  

\begin{equation*}
CH_{4} + 2O_{2} \rightarrow CO_{2} + 2H_{2}O_{(\text{ж})},\ \ Q_{\text{с}\text{г}}^{\text{в}\text{ы}\text{с}\text{ш}}\left( CH_{4} \right) = 890\ \text{к}\text{Д}\text{ж}/\text{м}\text{о}\text{л}\text{ь} \quad(1)
\end{equation*}

\begin{quote}\slshape\noindent
 А низшая теплота сгорания этого вещества - реакции:   $CH_{4} + 2O_{2} \rightarrow CO_{2} + 2H_{2}O_{(\text{г})},\ \ Q_{\text{с}\text{г}}^{\text{н}\text{и}\text{з}\text{ш}}\left( CH_{4} \right) = 802\ \text{к}\text{Д}\text{ж}/\text{м}\text{о}\text{л}\text{ь}$   
\end{quote}

\begin{enumerate}\keepwithnext
    \setcounter{enumi}{1}
    \item Вычислите молярную теплоту испарения воды   
    \setcounter{enumi}{2}
    \item Приведите формулы трёх веществ, для которых не будет наблюдатьс  разницы между низшей и высшей теплотами сгорания   Стандартные теплоты образования ацетилен  (C$_{2}$H$_{2}$), углекислого газа и жидкой воды равн  -227, 396 и 286 кДж/моль, соответственно   
    \setcounter{enumi}{3}
    \item Вычислите высшую и низшую теплоту сгорания ацетилена (на 1 мол  C$_{2}$H$_{2}$). Высшая мольная теплота сгорани  газообразного гексана C$_{6}$H$_{14}$ на 7,9 % превышае  низшую теплоту сгорания   
    \setcounter{enumi}{4}
    \item Запишите уравнение реакции сгорания гексана и вычислите высшую   низшую теплоту сгорания этого вещества. Удельные высшая и низшая теплоты сгорания некоторого углеводород  состава C$_{x}$H$_{y}$ составляют 46,91 кДж/г и 44,3  кДж/г соответственно  
    \setcounter{enumi}{5}
    \item Установите простейшую формулу неизвестного углеводорода, приведите     уравнение реакции его сгорания и рассчитайте его высшую и низшую     теплоты сгорания в кДж/моль.  
\end{enumerate}

\ul{Задача 30 (ММО-2018 2 тур)}  

Энергию связи (Е) можно вычислить прямым способом, используя $\Delta$H0х реакции M(g) + aL(g) = MLa(g), если достаточно данных для термодинамического цикла. Это возможно при вычислении сродства к протону (Ap).   1.Вычислите Ap NH3(g) + H+(g) = NH4+(g) и Еср связи N-H в NH4+, если $\Delta$fH0 равны (кДж/моль): -46.0 (NH3), 664 (NH4+), 218 (H), 473 (N), а потенциал ионизации H = H+ + e- равен I = 13.6 эВ.  

Данных достаточно также и для Me(g) + aCO(g) = Me(CO)a(g), так как известны $\Delta$fH0(кДж/моль): -110 (CO), -732 (Me(CO)a), 424 (Me). В карбониле WС = 30.6 мас.%.   2.Расшифруйте Me и карбонил.   3.Вычислите Еср связи Me-CO.  

Источники  

\begin{enumerate}\keepwithnext
    \setcounter{enumi}{0}
    \item <https://olymp.msu.ru/rus/page/main/29/page/zadaniya-olimpiady-proshlyh-let>  
    \setcounter{enumi}{1}
    \item <https://mathprofi.com/uploads/files/4518\textit{f}41_teoreticheskaya-i-matematicheskaya-himiya.pdf?key=345fc288eaa716eae5cc4b1b060509af>  
    \setcounter{enumi}{2}
    \item <https://olimpiada.ru/>  
    \setcounter{enumi}{3}
    \item https://www.chem.msu.ru/rus/olimp/  
\end{enumerate}