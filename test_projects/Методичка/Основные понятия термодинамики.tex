\section{Основные понятия термодинамики}

Химическая термодинамика - один из трех основных разделов физической химии, который изучает взаимные переходы различных форм энергий. Именно она позволяет качественно ответить на один из фундаментальных вопросов химической науки в целом: «Пойдет ли данная реакция при заданных условиях?», в случае утвердительного ответа на который следует дальнейшее изучение выбранного процесса, например, с точки зрения химической кинетики.  

Одним из основополагающих понятий химической термодинамики является термодинамическая система. Это некий материальный объект, который имеет реальную или воображаемую границу и способен обмениваться с окружающей средой (все, что вне термодинамической системы) энергией и (или) веществом. Существует три основных типа термодинамических систем \cref{fig:345a82}:  

а) Открытая - система, способная обмениваться как веществом, так и энергией. Наиболее сложный тип термодинамических систем, ярким примером является человек;  

б) Закрытая - система, способная обмениваться энергией, но не веществом, например, пробирка, закрытая резиновой пробкой;  

в) Изолированная - система, неспособная на обмен энергией и веществом. Это идеальный и наиболее простой тип систем, к которому будет сводиться любой объект в олимпиадных задачах. В реальной жизни довольно трудно найти пример изолированной системы, наиболее приближенным, пожалуй, можно считать термос.  

\begin{figure}[H] 
\centering
\includegraphics[width = \textwidth, keepaspectratio]{./images/image1v.png}
\caption{Типы термодинамических систем: а) открытая; б) закрытая; в) изолированная.}
\label{fig:345a82}
\end{figure}

Для количественной характеризации термодинамической системы используют различные термодинамические переменные (свойства). Чаще всего их делят на две основные группы по зависимости от количества вещества:  

\begin{itemize}\keepwithnext
    \item Экстенсивные - прямо пропорциональны массе системы и числу частиц. К ним относят объем (V), внутреннюю энергию (U), энтропию (S) и тд.  
    \item Интенсивные - не зависят от массы системы и числа частиц, например, температура (T), плотность ($\rho$), давление (p) и тд.  
\end{itemize}

Важно помнить, что отношение любых двух экстенсивных параметров является интенсивным параметром. Данное свойство довольно легко проверить, вспомнив определение плотности ($\rho = m/V$).  

Любое изменение в термодинамической системе, связанное с изменением хотя бы одной из ее термодинамических переменных называют термодинамическим процессом. Существуют самопроизвольные (без затрат Е), несамопроизвольные (с затратами Е), обратимые, квазистатические, необратимые и многие другие процессы. В зависимости от фиксации той или иной переменной их можно разделить на следующие:  

\begin{enumerate}\keepwithnext
    \setcounter{enumi}{0}
    \item Изотермический (T=const);  
    \setcounter{enumi}{1}
    \item Изохорный (V=const);  
    \setcounter{enumi}{2}
    \item Изобарный (p=const);  
    \setcounter{enumi}{3}
    \item Адиабатический (Q=0 или $\delta$Q=0).  
\end{enumerate}

Вся химическая термодинамика базируется на двух постулатах и трех законах (будут рассмотрены позже). Основной постулат термодинамики ограничивает диапазон исследуемых данной наукой объектов и утверждает отсутствие времени в ней. Звучит он следующим образом:  

\begin{quote}\slshape\noindent
«Любая изолированная система с течением времени приходит в равновесное состояние и самопроизвольно не может из него выйти.»  
\end{quote}

Второй постулат термодинамики вводит понятие температуры как интенсивной величины и задает аксиомой термическое равновесие, из которого следуют все остальные.  

\begin{quote}\slshape\noindent
«Если система А находится в тепловом равновесии с системой В, а та в свою очередь с системой С, то А и С также находятся в тепловом равновесии.»  
\end{quote}