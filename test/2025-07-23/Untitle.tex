Вот задача, оформленная в стиле школьной олимпиады по неорганической химии, с акцентом на умение написать формулу гетерополисоединения:

---

### Условие:

Гетерополисоединения — это кислородсодержащие комплексы переходных металлов с анионами, содержащими гетероатомы, такие как фосфор, кремний, мышьяк и др. Их синтез основан на реакции поликонденсации молибдатов или вольфраматов с гетероанионами в кислой среде.

Один из таких примеров — образование фосфомолибдата аммония. При добавлении раствора фосфорной кислоты к молибденовой жидкости (раствору гептамолибдата аммония в азотной кислоте) наблюдается образование жёлтого осадка — соединения B. Эта реакция применяется в аналитической химии для качественного определения фосфат-ионов:

> 7H3PO4+12(NH4)6Mo7O24+51HNO3⟶7(NH4)3[PMo12O40]⋅2H2O↓+51NH4NO3+22H2O7 \text{H}_3\text{PO}_4 + 12 (NH_4)_6\text{Mo}_7\text{O}_{24} + 51 \text{HNO}_3 \longrightarrow 7 (NH_4)_3[\text{PMo}_{12}\text{O}_{40}] \cdot 2H_2O \downarrow + 51 NH_4NO_3 + 22 H_2O

---

### Вопрос:

Напишите формулу соединения B, образующегося при описанной реакции. Ответ запишите в виде химической формулы, например, NaCl\text{NaCl}.

> _Критерии оценивания: за правильный ответ — 5 баллов._

---

**Ответ:** (NH4)3[PMo12O40]⋅2H2O(NH_4)_3[PMo_{12}O_{40}] \cdot 2H_2O

---

Если хочешь, могу подготовить аналогичную задачу на кремний или мышьяк, чтобы показать вариативность по гетероатому.